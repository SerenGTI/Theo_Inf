\chapter{Grundlagen der Aussagenlogik}
\section{Syntax der Aussagenlogik}
\begin{itemize}
	\item Atomare Formeln: $A_i$ mit $i\in\N$
	\item $F$ und $G$ Formeln $\rightarrow$ $(F\wedge G)$, $(F\vee G)$ und $\neg F$ auch Formeln
\end{itemize}
\section{Semantik der Aussagenlogik}
$D\subseteq\simpleset{A_1,A_2,\ldots}$

Eine Abbildung $\mathcal A:D\rightarrow\simpleset{0,1}$ heißt Belegung.
Weiter

\begin{itemize}
	\item Eine Belegung $\mathcal A$ ist \textbf{passend} zu einer  Formel $F$, falls alle in $F$ vorkommenden atomaren Variablen zum Definitionsbereich von $\mathcal A$ gehören.
	\item Eine Belegung $\mathcal A$ ist \textbf{Modell} für eine Formel, falls $\mathcal A$ zu $F$ passend ist und $\mathcal A(F)=1$ gilt.\\
			Man schreibt dann $\mathcal A \vDash F$.
	\item $F$ ist \textbf{erfüllbar}, falls ein Modell für $F$ existiert.
	\item $F$ ist \textbf{gültig}, falls alle passenden Belegungen Modelle sind. $\rightarrow$ $F$ nennt man dann eine  \textbf{Tautologie}. Das Komplement einer Tautologie ist unerfüllbar.
	\item Zwei Formeln $F$ und $G$ heißen \textbf{semantisch äquivalent}, wenn alle zu beiden passenden Belegungen $\mathcal A$ gilt: $\mathcal A(F) = \mathcal A(G)$. Man schreibt dann $F\equiv G$.
\end{itemize}

\section{Prädikatenlogik}
