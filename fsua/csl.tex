\chapter{Kontextsensitive Sprachen $\csl$: Typ-1}
\begin{equation*}
	\csl\subset\rec\subset\re
\end{equation*}
Wortproblem entscheidbar. Abschluss unter allen Operationen.

\section{Algorithmus zur Entscheidbarkeit des Wortproblems:}
Die Funktion $\mathrm{Abl}_n(X)$ wird iteriert angewendet, bis sich entweder $X$ nicht mehr ändert ($w\not\in L(G)$) oder das gesuchte Wort $w$ in $X$ enthalten ist ($w\in L(G)$).

Dabei ist $n$ die Länge des gesuchten Worts $w$, also $|w|$.

Die Funktion $\mathrm{Abl}_n(X)$ ist für eine Grammatik $G$ wie folgt definiert:
\begin{equation*}
	\mathrm{Abl}_n(X)\coloneqq X\cup\set{w\in(V\cup\Sigma)^\star}{|w|\leq n \wedge \exists y\in X:y\Rightarrow_G w}
\end{equation*}


\section{Automatenmodell LBA:}
Die Kontextsensitiven Sprachen werden von linear beschränkten Turingmaschinen akzeptiert.



\section{Beispiele: }
\begin{itemize}
	\item $L_1=\set{a^nb^nc^n}{n\geq 1}$ (drei und mehr gleiche Exponenten)
	\item
\end{itemize}
