\chapter{Rekursiv aufzählbare Sprachen}\label{sec:typ0}
\begin{equation*}
	\operatorname{Typ-0} = \re
\end{equation*}



Eine Turingmaschine $M$ \emph{akzeptiert} die Sprache $L$, wenn sie nach endlicher Zeit hält.
Bei Eingaben, die nicht zu $L$ gehören, rechnet sie unendlich lang weiter (dieses Verhalten führt später auf das Halteproblem).
Solche Sprachen $L$ gehören dann zu $\re$, sie sind \emph{semi-entscheidbar}.



\section{Sätze zu $\re$}
\begin{itemize}
	\item Die Klasse der Typ-0 Sprachen ist abgeschlossen unter Sternopertaion, Vereinigung, Schnitt und Konkatenation.
	\item Die Klasse der durch Turingmaschinen erkennbaren/akzeptierten Sprachen ist gleich der Klasse der rekursiv aufzählbaren.
\end{itemize}
