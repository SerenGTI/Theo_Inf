%!TEX root = ../main.tex
\chapter{Entscheidbarkeitstheorie}
Die Berechenbarkeitstheorie spielt sich innerhalb der Entscheidbarkeitstheorie ab. Wenn eine Sprache entscheidbar ist, macht es Sinn ihre Komplexität zu bestimmen. Andersherum heißt das, dass jede Sprache, die in einer der Komplexitätsklassen enthalten ist entscheidbar ist (also insbesondere semi- und co-semi-entscheidbar).

\section{Definitionen}
\subsection{Charakteristische Funktionen}
Für jede Menge $A$ existiert die sogenannte \emph{charakteristische Funktion} $\chi_A(w)$. Diese Funktion entscheidet für jedes Wort $w$ aus einer festgelegten Grundmenge, ob $w\in A$ gilt.
\begin{equation*}
	\chi_A(w)=\begin{cases}
		1\text{, falls } w\in A\\
		0\text{, sonst}
	\end{cases}
\end{equation*}

Ebenso lässt sich die semi-charakteristische Funktion $\chi_A'(w)$ definieren
\begin{equation*}
	\chi_A'(w)=\begin{cases}
		1\text{, falls } w\in A\\
		\mathrm{undefiniert}\text{, sonst}
	\end{cases}
\end{equation*}

Eine Menge heißt \emph{entscheidbar}, wenn ihre zugehörige charakteristische Funktion \emph{berechenbar} ist. Eine Menge heißt semi-entscheidbar, falls die semi-charakteristische Funktion berechenbar ist. Das Komplement einer Mengen, deren semi-charakteristische Funktion berechenbar ist, heißt co-semi-entscheidbar.

Ist eine Menge entscheidbar, so ist es ihr Komplement trivialerweise auch.

\subsection{Rekursive Aufzählbarkeit}
Eine Sprache $A$ ist rekursiv aufzählbar, wenn eine totale, berechenbare Funktion $c:\N\rightarrow\Sigma^\star$ existiert sodass $A=\set{c(n)}{n\in\N}$ gilt. Rekursive Aufzählbarkeit ist äquivalent zu Semi-Entscheidbarkeit.

\subsection{Entscheidbarkeitsprobleme}
Hierbei seien die Gödelisierungen der Maschinen in einer geeigneten Art und Weise kodiert. Beispielsweise $w\in\simpleset{0,1}^\star$ für eine Maschine $M_w$.
\medskip

\begin{tabular}{r|c|l}
	Spezielles Halteproblem & $K=\set{w}{M_w \text{ hält auf Eingabe } w}$ & semi-entscheidbar\\
	Allgemeines Halteproblem & $H=\set{w\# x}{M_w \text{ hält auf Eingabe } x}$ & semi-entscheidbar\\
	Halteproblem auf leerem Band & $H_0=\set{w}{M_w \text{ hält auf Eingabe } \epsilon}$ & semi-entscheidbar\\
	PCP & Postsches Korrespondenzproblem & semi-entscheidbar\\
	MPCP & Für alle Lösungen gilt $i_1=1$ & semi-entscheidbar\\
	WA & Menge aller wahren arithmetischen Formeln & unentscheidbar\\
	$\overline{\text{WA}}$ & Menge aller falschen arithm. Formeln & unentscheidbar
\end{tabular}

\paragraph{Satz von Rice}
Sei $\mathcal R$ die Klasse der Turing-berechenbaren Funktionen und $\mathcal S$ eine nichttriviale Teilmenge von $\mathcal R$.
Dann ist die Menge $C(\mathcal S)=\set{w}{M_w \text{ berechnet eine Funktion aus } \mathcal S}$ unentscheidbar.

Wichtig ist, dass der Satz von Rice nur Aussagen über berechnete Funktionen macht, Aussagen über die berechnende Turingmaschine werden nicht in Betracht gezogen.

Bei der Verwendung des Satzes von Rice führt man eigentlich eine \hyperref[sec:reduktion]{Reduktion} durch, muss sich jedoch keine Funktion mehr einfallen lassen.

\paragraph{Probleme in der Theorie der formalen Sprachen}
\begin{itemize}
	\item Für deterministisch kontextfreie Sprachen $L,K\in\dcfl$ sind die Fragen $L\cap K=\emptyset$, $|L\cap K|<\infty$, $L\cap K \in \cfl$ sowie $L\subseteq K$ unentscheidbar.
	\item Für eine kontextfreie Grammatik sind die Frage nach Mehrdeutigkeit, $\overline{L(G)}\in\cfl$, $L(G)\in \reg$ und $L(G)\in \dcfl$ alle unentscheidbar.
	\item Für eine kontextsensitive-Grammatik ist die Leerheit sowie die Endlichkeit der erzeugten Sprache unentscheidbar.
\end{itemize}

\section{Reduktion von Problemen}\label{sec:reduktion}
Um die prinzipielle Lösbarkeit zweier Probleme zu betrachten, verwendet man Reduktionen:

Seien $A\subseteq \Sigma^\star$ und $B\subseteq \Gamma^\star$ zwei Probleme. Kann man eine \emph{totale und berechenbare Funktion} $f:\Sigma^\star\rightarrow\Gamma^\star$ finden, so dass gilt
\begin{equation*}
	x\in A\Leftrightarrow f(x)\in B,
\end{equation*}
so sagt man, $A$ ist auf $B$ (many-one-)reduzierbar.
Man schreibt dann auch $A\leq B$.

\begin{enumerate}
	\item $A\leq B$ und $B$ entscheidbar $\Rightarrow$ $A$ entscheidbar.
	\item $A\leq B$ und $B$ semi-entscheidbar $\Rightarrow$ $A$ semi-entscheidbar.
\end{enumerate}

Bei der Reduktion übertragen sich immer fehlende Eigenschaften von links nach rechts, d.h. reduziert man ein semi-entscheidbares, aber nicht co-semi-entscheidbares Problem $A$ auf ein Problem $B$ (d.h. $A\leq B$), so ist $B$ ebenfalls nicht co-semi-entscheidbar. Über die Semi-Entscheidbarkeit von $B$ wird keine Aussage gemacht.

Reduziert man allerdings ein unbekanntes Problem $A$ auf ein semi-entscheidbares Problem $B$ (d.h. $A\leq B$), so hat man die Semi-Entscheidbarkeit von $A$ gezeigt. Es übertragen sich also \glqq positive Eigenschaften\grqq\  von rechts nach links.

Letzterer Sachverhalt wird klar, wenn man sich überlegt, dass man mit der Reduktion eine Funktion gefunden hat, die eine Fragestellung aus $A$ in eine aus $B$ überträgt. Man kann also das Entscheidungsproblem für $A$ zunächst in $B$ übersetzen und schließlich mit dem Entscheidungsalgorithmus für $B$ entscheiden.

\paragraph{Beispiel einer Reduktion:}
Wir wollen zeigen, dass die Sprache
$$L=\set{w\in\simpleset{0,1}^\star}{T(M_w)\in\reg}$$
das heißt die Menge der Turingmaschinen, die reguläre Sprachen akzeptieren, unentscheidbar ist. Dafür reduzieren wir $H_0\leq L$, dabei sei $f(w)$ die Kodierung einer Turingmaschine die auf Eingabe $x$ wie folgt arbeitet
\begin{itemize}
	\item Wenn die Eingabe von der Form $x=a^nb^n$ mit $n\in\N$ ist, akzeptiere.
	\item Sonst lösche die Eingabe und simuliere $M_w$ auf leerem Band.
\end{itemize}
Hierbei sei die Maschine o.B.d.A. über einem Alphabet $\simpleset{a,b}\subset\Sigma$ definiert.
Damit ist $f$ total und berechenbar und es gilt
\begin{align*}
	w\in H_0&\Rightarrow T(M_{f(w)})=\Sigma^\star & w\not\in H_0&\Rightarrow T(M_{f(w)})=\set{a^nb^n\in\Sigma^\star}{n\in\N}\\
	&\Rightarrow T(M_{f(w)})\in\reg&&\Rightarrow T(M_{f(w)})\in\cfl\setminus\reg 
\end{align*}
Da $H_0$ unentscheidbar ist, ist es auch $L$.\hfill$\Box$