\chapter{Deterministisch Kontextfreie Sprachen $\dcfl$:}
\begin{equation*}
	\dcfl\subset\cfl\subset\csl\subset\rec\subset\re
\end{equation*}
Wort-, Leerheits- und Äquivalenzproblem entscheidbar. Abschluss nur unter Komplement.

Beschreib- und erkennbar durch einen endlichen deterministischen Kellerautomaten (DPDA).



\section{Automatenmodell DPDA:}\label{dcfl:dpda}


\section{Sätze zu den deterministisch kontextfreien Sprachen:}
\begin{itemize}
	\item Eine deterministisch kontextfreie Sprache geschnitten mit einer regulären Sprache ist wieder eine deterministisch kontextfreie Sprache.
	\begin{equation*}
		L_1\in\dcfl, L_2\in\reg \Rightarrow L_1\cap L_2\in\dcfl
\end{equation*}
\end{itemize}

\section{Beispiele: }
\begin{itemize}
	\item $L_1=\set{w\$w^R}{w\in\Sigma^\star}$ (markierte Palindrome)
\end{itemize}
