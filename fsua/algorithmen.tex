\chapter{Algorithmen}
\section{Konstruktionsalgorithmus für Minimal-DEAs:}
Mit dem Beweis zur Myhill-Nerode-Äquivalenz wird ein Automat definiert, dieser ist isomorph zum Minimalautomaten.
Der Index der Myhill-Nerode-Äquivalenz ist genau die Anzahl der Zustände des Minimalautomaten.

Man kann mit einem einfachen Algorithmus aus einem beliebigen DEA den Minimalautomaten erzeugen:

Wir ermitteln algorithmisch, welche Zustände nicht äquivalent sind und verschmelzen die übrig bleibenden.
Nicht äquivalent sind Zustände, bei denen bei Eingabe eines Worts vom einen aus ein Endzustand erreicht wird, vom anderen jedoch nicht.

So sind im ersten Schritt Zustandspaare aus Endzustand und Nichtendzustand nicht äquivalent und werden markiert.

\subsection*{Am Beispiel:}

\vspace{1em}
\begin{tikzpicture}[->,>=stealth',shorten >=1pt,auto,node distance=2.8cm,
										semithick]
	\tikzstyle{every state}=[fill=none,draw=black,text=black]

	\node[initial,state] 						(A)              {$z_0$};
	\node[state]         						(B) [below of=A] {$z_1$};
	\node[state]         						(C) [below right of=A] {$z_2$};
	\node[state]         						(D) [right of=A] {$z_3$};
	\node[state, accepting]         (E) [right of=C] {$z_4$};

	\path (A) edge 							node {a} (B)
						edge			  			node {b} (C)
				(B) edge [bend right] node {a} (E)
						edge 						  node {b} (C)
				(C) edge 						  node {a} (D)
						edge [loop right] node {b} (C)
				(D) edge 						  node {a} (E)
						edge 							node {b} (A)
				(E) edge [loop right] node {a,b} (E);
\end{tikzpicture}


Die Paare $\simpleset{z_i,z_4}$ mit $i=0,1,2,3$ werden markiert:

\begin{tabular}{ccccc}
	\cline{2-2}
	$z_1$ & 	\multicolumn{1}{|c|}{ }		&				&				&			 \\
	\cline{2-3}
	$z_2$ & 	\multicolumn{1}{|c|}{ }		&		\multicolumn{1}{c|}{ }		&				&			 \\
	\cline{2-4}
	$z_3$ & 	\multicolumn{1}{|c|}{ }		&		\multicolumn{1}{c|}{ }		&		\multicolumn{1}{c|}{ }		&			 \\
	\cline{2-5}
	$z_4$ & 	\multicolumn{1}{|c|}{$\epsilon$}		&		\multicolumn{1}{c|}{$\epsilon$}		&		\multicolumn{1}{c|}{$\epsilon$}		&		\multicolumn{1}{c|}{$\epsilon$}	 \\
	\cline{2-5}
				& $z_0$ & $z_1$ & $z_2$ & $z_3$\\
\end{tabular}

Durch Testen der Zustandspaare erhält man die untenstehende Tabelle. Hierbei wurden Zeugen für die Inäquivalenz eingetragen.

\begin{tabular}{ccccc}
	\cline{2-2}
	$z_1$ & 	\multicolumn{1}{|c|}{a}		&				&				&			 \\
	\cline{2-3}
	$z_2$ & 	\multicolumn{1}{|c|}{ }		&		\multicolumn{1}{c|}{a}		&				&			 \\
	\cline{2-4}
	$z_3$ & 	\multicolumn{1}{|c|}{a}		&		\multicolumn{1}{c|}{ }		&		\multicolumn{1}{c|}{a}		&			 \\
	\cline{2-5}
	$z_4$ & 	\multicolumn{1}{|c|}{$\epsilon$}		&		\multicolumn{1}{c|}{$\epsilon$}		&		\multicolumn{1}{c|}{$\epsilon$}		&		\multicolumn{1}{c|}{$\epsilon$}	 \\
	\cline{2-5}
				& $z_0$ & $z_1$ & $z_2$ & $z_3$\\
\end{tabular}

Damit lassen sich Zustände zusammenfassen: $p=\simpleset{z_0,z_2}, q=\simpleset{z_1,z_3}$. Der Minimalautomat ist also:

\vspace{1em}
\begin{tikzpicture}[->,>=stealth',shorten >=1pt,auto,node distance=3cm,
										semithick]
	\tikzstyle{every state}=[fill=none,draw=black,text=black]

	\node[initial,state] 						(A)              {$p$};
	\node[state]         						(B) [right of=A] {$q$};
	\node[state, accepting]         (C) [right of=B] {$z_4$};

	\path (A) edge [bend left]	node {a} (B)
						edge [loop above]	node {b} (A)
				(B) edge  						node {a} (C)
						edge [bend left]  node {b} (A)
				(C) edge [loop right] node {a,b} (C);
\end{tikzpicture}

\begin{equation*}
	T(M)=\set{waaw'}{w,w'\in\simpleset{a,b}^\star}
\end{equation*}


\section{CYK-Algorithmus zur Lösung des Wortproblems für Typ-2}\label{algo:cyk}
Mit dem CYK-Algorithmus ist das Wortproblem für Typ-2 in $\mathcal O(n^3)$ entscheidbar.

Hierfür werden alle Ableitungsmöglichkeiten in einer Tabelle geordnet dargestellt. Ist am Ende die Startvariable als Startknoten für die Ableitung möglich, so ist das Wort in $L$.


\subsection*{Am Beispiel:}
Produktionsregeln der Grammatik (CNF):

\begin{equation*}
	S\rightarrow AX|YB, A\rightarrow XA|AB|a, B\rightarrow XY|BB, X\rightarrow YA|a, Y\rightarrow XX|b
\end{equation*}

Eingabewort: $abbaab$

\renewcommand{\arraystretch}{1.2}
\begin{tabular}{ccccccc}
	Länge & a & b & b & a & a & b\\
	\cline{2-7}
	1 & \multicolumn{1}{|c|}{$\simpleset{A,X}$} & \multicolumn{1}{c|}{$\simpleset{Y}$} & \multicolumn{1}{c|}{$\simpleset{Y}$} & \multicolumn{1}{c|}{$\simpleset{A,X}$} & \multicolumn{1}{c|}{$\simpleset{A,X}$} & \multicolumn{1}{c|}{$\simpleset{Y}$}\\
	\cline{2-7}
	2 & \multicolumn{1}{|c|}{$\simpleset{B}$} & \multicolumn{1}{c|}{$\emptyset$} & \multicolumn{1}{c|}{$\simpleset{X}$} & \multicolumn{1}{c|}{$\simpleset{A,S,Y}$} & \multicolumn{1}{c|}{$\simpleset{B}$} &\\
	\cline{2-6}
	3 & \multicolumn{1}{|c|}{$\emptyset$} & \multicolumn{1}{c|}{$\emptyset$} & \multicolumn{1}{c|}{$\simpleset{A,Y,X}$} & \multicolumn{1}{c|}{$\simpleset{A}$} &&\\
	\cline{2-5}
	4 & \multicolumn{1}{|c|}{$\emptyset$} & \multicolumn{1}{c|}{$\simpleset{X}$} & \multicolumn{1}{c|}{$\simpleset{B,X}$} &&&\\
	\cline{2-4}
	5 & \multicolumn{1}{|c|}{$\simpleset{S,Y}$} & \multicolumn{1}{c|}{$\simpleset{B,S}$} &&&&\\
	\cline{2-3}
	6 & \multicolumn{1}{|c|}{$\simpleset{A,B}$} &&&&&\\
	\cline{2-2}
\end{tabular}

Da die unterste Zelle nun $\simpleset{A,B}$ enthält, ist $w=abbaab\not\in L$
