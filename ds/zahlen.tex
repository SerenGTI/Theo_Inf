%!TEX root = ../main.tex
\chapter{Wichtige Abschätzungen}

\section{Wachstum der Fakultät}
\begin{equation*}
	\log(n!)\in \Theta(n\log n)
\end{equation*}
denn es gilt 
\begin{align*}
	\forall n\geq 2:\quad \left(\frac n2\right)^{\frac n2} < n! < n^n \text{ oder auch } e*\left(\frac ne\right)^n\leq n!\leq n*e*\left(\frac ne\right)^n
\end{align*}

\section{Wachstum des Binomialkoeffizienten}
Wir interessieren uns für die Binomialkoeffizienten der Form $\binom{2n}{n}$ bzw. $\binom{n}{\lceil \frac n2\rceil}=\binom{n}{\lfloor \frac n2\rfloor}$ (diese sind die Binomialkoeffizienten mit größtem Wert) für große $n$.
Aus dem binomischen Lehrsatz folgt (mit $a=b=1$), dass
\begin{equation*}
	\sum_k\binom nk = 2^n.
\end{equation*}
Damit wissen wir, dass die Binomialkoeffizienten im Durchschnitt von der Größenordnung $\frac{2^n}n$ sind. Damit folgt
\begin{equation*}
	\forall n\geq 3:\quad\binom{n}{\lceil \frac n2\rceil}=\binom{n}{\lfloor \frac n2\rfloor}>\frac{2^n}n
\end{equation*}



\section{Wachstum des kleinsten gemeinsamen Vielfachen}
\newcommand{\kgV}{\texttt{kgV}}
Wir definieren zunächst
\begin{equation*}
	\kgV(n)=\kgV(2,\ldots,n).
\end{equation*}
es ist also das kleinste gemeinsame Vielfache der ersten $n$ natürlichen Zahlen.

\section{Fibonacci-Zahlen}
\begin{equation*}
	F_0 = 0, F_1=1, F_n=F_{n-1}+F_{n-2}\enspace\forall n\geq 2
\end{equation*}

Sogenannte Dominostein-Interpretation der Fibonacci-Zahlen:
Es stehen beliebig viele Dominosteine von zwei Sorten zu Verfügung, solche der Länge 1 und solche der Länge 2.

Wie viele Möglichkeiten gibt es, damit eine Sequenz der Länge $n$ zu legen? $\rightsquigarrow F_n$

\subsection{Abschätzung der Fibonacci-Zahlen}
Es gilt 
\begin{align*}
	&F_n\leq 2^n\leq F_{2n}\\
	\text{bzw. }&2^n\leq F_{2n}\leq 2^{2n}\\
	\text{bzw. }&(\sqrt 2)^n\leq F_n\leq 2^n
\end{align*}
dies lässt sich durch Induktion leicht zeigen.

Es gilt
\begin{equation*}
	\ggT(F_m,F_n)=F_{\ggT(m,n)}.
\end{equation*}



\section{Catalanzahlen}

