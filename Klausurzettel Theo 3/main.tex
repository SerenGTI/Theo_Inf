\documentclass[landscape, 8pt, a4paper]{extarticle}
\usepackage[ngerman]{babel}
\usepackage[utf8]{inputenc}
\usepackage{lmodern}
\usepackage[T1]{fontenc}
\renewcommand{\familydefault}{\sfdefault}

\usepackage{amsmath}
\usepackage[fleqn]{mathtools}
\usepackage{amssymb}
\usepackage{amsfonts}
\usepackage{latexsym}
\usepackage[landscape, margin=1cm]{geometry}
\usepackage{enumitem}
\usepackage{multicol}
\usepackage{fancyhdr}
\usepackage{xcolor}


\mathcode`\*="8000 %Sterne durch Malpunkte ersetzen
{\catcode`\*\active\gdef*{\cdot}}

\pagestyle{empty}

\setlength{\parindent}{0pt}

\newcommand{\ggT}{\texttt{ggT}}
\newcommand{\kgV}{\texttt{kgV}}


\setlist[2]{noitemsep}
\setitemize{noitemsep, leftmargin=8pt, itemindent=0pt, labelsep=3pt, labelwidth=0pt, labelindent=0pt}
\setlength{\belowdisplayskip}{0pt} \setlength{\belowdisplayshortskip}{0pt}
\setlength{\abovedisplayskip}{0pt} \setlength{\abovedisplayshortskip}{0pt}

\newcommand{\N}{\mathbb{N}}
\newcommand{\Z}{\mathbb{Z}}
\newcommand{\set}[2]{\ensuremath\left\{ #1 \,\middle|\, #2 \right\}}
\newcommand{\simpleset}[1]{\ensuremath\left\{ #1 \right\}}
\renewcommand{\O}{\mathcal O}

\begin{document}
\begin{multicols}{3}
\raggedbottom
	\textbf{Simon König 3344789 - Klausurzettel Theoretische Informatik 3}

	\section{Wichtige Algorithmen}
	\begin{itemize}
		\item \textbf{Quickselect} wählt das $k$ kleinste Element aus einer unsortierten Liste aus. Laufzeit w-c: $\O(n^2)$ und a-c: $\leq4n$
		\item \textbf{Quicksort} $\O(n\log n)$ worst-case:$\O(n^2)$, \textbf{Merge- und Heapsort} $\O(n\log n)$, \textbf{Dijkstra} $\O(n^2+m)$. \textbf{Bottom-Up Heapsort} in $(1.5*n\log n)$ und \textbf{Ultimatives Heapsort} $(n\log n) + \O(n)$

		\item Algorithmus zur \textbf{optimalen Klammerung} (dynamische Programmierung). Laufzeit $\O(n^3)$
		Kosten für Matrixmultiplikation $(k\times m)*(m\times l)$ sind $k*m*l$.

		Tabelleneintrag $T_{i,j}$ enthält die minimalen Kosten die Matrizen $M_i$ bis $M_j$ zu multiplizieren.

		$T_{i,j}=\min_{i\leq m\leq j}\simpleset{T_{i,m}+T_{m+1,j}+n_{i-1}*n_{m}*n_{j}}$.

		Tabelle $B$ enthält die Trennpunkte, $T_{1,3}=1$ bedeutet, dass die Trennstelle nach Matrix $2$ kommt, d.h. $M_1(M_2M_3)$.

		\item \textbf{Schnelle Multiplikation} von $X=A*b^n+B$ und $Y=C*b^n+D$
		\begin{equation*}
			XY=AC*b^{2n}+(AD+BC)*b^n+BD
		\end{equation*}
		wir berechnen
		\begin{align*}
			P_1&=AC\\
			P_2&=BD\\
			P_3&=(A+B)(C+D)=AC+AD+BC+BD\\
			&\rightarrow \quad P_3-P_1-P_2=(AD+BC).
		\end{align*}
		damit $o(n^{1.6})$.

		\item Randomisierte Algorithmen: Monte Carlo, Fehler ist erlaubt (Primzahltest) und Las Vegas, Kein Fehler erlaubt aber keine Aussage ist auch möglich (Quicksort). 

		\item \textbf{Primzahltest nach Fermat} Wähle $a\in\simpleset{1,\ldots,n}$, $x=a^{n-1}\mod n$. Wenn $x\not\equiv 1\mod n$, dann ist $n$ keine Primzahl. Sonst keine Aussage.

		\item \textbf{Primzahlzertifikat} Sei $n\geq 2$ und $n\in\N$. Falls für alle Primzahlen $p$ mit $n\equiv 1\mod p$ eine Zahl $a\in Z$ existiert, so dass
		\begin{equation*}
		 	a^{n-1}\equiv 1\mod n \text{\quad und \quad}a^{\frac{n-1}p}\not\equiv 1\mod n
		\end{equation*}
		gilt, dann ist $n$ eine Primzahl. 

		\item \textbf{RSA}
		\begin{enumerate}
			\item Große Primzahlen $p,q$ mit $p<q$
			\item Berechne $n=p*q$
			\item setze $\varphi(n)=(p-1)(q-1)$,
			\item wähle $e>1$ mit $\ggT(e,\varphi(n))=1$
			\item Berechne $s<n$ mit $e*s\equiv 1\mod \varphi(n)$, d.h. $e*s=k*\varphi(n)+1$. 
		\end{enumerate}
		Das Paar $(n,e)$ bildet den \emph{öffentlichen Schlüssel}. Die Nachricht sei $x$. Die verschlüsselte Nachricht ist dann $y$
		\begin{equation*}
			y=x^e\mod n\quad x'=y^s\mod n
		\end{equation*}

		\item \textbf{String-Matching} 
	\end{itemize}


	\section{Laufzeitanalyse}
	\begin{itemize}
		\item \textbf{Mastertheorem 1} Für $a,b\in\N$, $b>1$ und eine Funktion $g:\N\rightarrow \N$ mit $g\in\Theta(n^c)$ gelte
		\begin{align*}
			t(1)&=g(1)\\
			t(n)&=a*t\left(\frac nb\right)+g(n)
		\end{align*}
		Dann gilt 
		\begin{equation*}
			t(n)\in\begin{cases}
			\Theta(n^c)&\text{falls }a<b^c\\
			\Theta(n^c\log n)&\text{falls }a=b^c\\
			\Theta(n^{\frac{\log a}{\log b}})&\text{falls }a>b^c\\
			\end{cases}
		\end{equation*}

		\item \textbf{Mastertheorem 2} Sei $r>0$ und die Zahlen $\alpha_1\geq 0$ für alle $i$ und erfüllen $\sum_{i=1}^r\alpha_i<1$.
		Wenn die Rekursive Funktion $t$ die Ungleichung
		\begin{equation*}
			t(n)\leq \left( \sum_{i=1}^r t(\lceil \alpha_i*n\rceil) \right)+c*n
		\end{equation*}
		für ein $c>0$ erfüllt, dann ist $t(n)\in\O(n)$.

		\item \textbf{Summenformeln}
		\begin{align*}
			\sum_{i=0}^n x^i &= \frac{1-x^{n+1}}{1-x}\\
			\sum_{i=1}^n i &= \frac{n^2+n}{2}
		\end{align*}

		\item \textbf{Landausymbole} Damit $f\in \Lambda(g)$ gilt

		\begin{tabular}{r|c|c}
			$\O$&$\limsup f/g<\infty$&$\exists c,N:\forall n\geq N: f(n)\leq c*g(n)$\\
			$o$&$\lim f/g=0$&$\forall c\exists N:\forall n\geq N: f(n)\leq c*g(n)$\\
			$\Omega$&$\liminf f/g>0$&$\exists c,N:\forall n\geq N: f(n)\geq c*g(n)$\\
			$\omega$&$\lim f/g=\infty$&$\forall c\exists N:\forall n\geq N: f(n)\geq c*g(n)$\\
			$\Theta$&$\O\cap\Omega$&\\
		\end{tabular}

		Man darf immer $f$ und $g$ in Betragsstriche setzen.
	\end{itemize}



	\section{Diskrete Strukturen}
	\begin{itemize}
		\item Halbgruppe: assoziativ
		\item Monoid: neutrales Element
		\item Gruppe: beidseitig Inverse
		\item Abelsche Gruppe: kommutativ
		\item Ring: $(R,+,0)$ ist abelsche Gruppe, $(R,*,1)$ ist Monoid, Distributivgesetze
		\item Körper: $(R\setminus\simpleset 0, *,1)$ ist Gruppe und $*$ ist kommutativ

		
		\item Eigenschaften einer Kongruenzrelation:%
		\begin{itemize}%
			\item Reflexivität: $a\sim a$
			\item Symmetrie: $a\sim b\Rightarrow b\sim a$
			\item Transitivität: $a\sim b, b\sim c\Rightarrow a\sim c$
			\item Kongruenzeigenschaft mit Abbildung $\circ$: $x\sim x'\wedge y\sim y'\Rightarrow x\circ y\sim x'\circ y'$ 
		\end{itemize}

	\end{itemize}


	\section{Modulare Arithmetik}
	\begin{itemize}
		\item Es gilt
		\begin{equation*}
			ca\equiv cb\mod m\Rightarrow a\equiv b\mod \frac{m}{\ggT(m,c)}
		\end{equation*}
		\item Möchte man $ax\equiv b\mod m$ lösen und man kann nicht durch $a$ teilen, kann man mit dem Inversen von $a$ multiplizieren. Man erhält dann $x\equiv a^{-1}b\mod m$.

		\item Ist bei der Kongruenz $ax\equiv b \mod m$ der $\ggT(a,m)$ kein Teiler von $b$, dann ist die Kongruenz nicht lösbar.

		\item Allgemein gilt
		\begin{equation*}
			\ggT(ca,cb)=c*\ggT(a,b)
		\end{equation*}
		\item \textbf{Lemma von Bezout}
		Für alle $m,n\in\Z$ existieren $a,b\in\Z$ so, dass
		\begin{equation*}
			\ggT(m,n)=am+bn
		\end{equation*}
		das heißt, der größte gemeinsame Teiler lässt sich als Linearkombination darstellen.
		\item \textbf{Fundamentalsatz der Arithmetik} Die Primfaktorzerlegung jeder natürlichen Zahl ist eindeutig.
		\item Die multiplikative Gruppe besteht aus den Elementen, die teilerfremd zum Modul sind
		\begin{equation*}
			(\Z/n\Z)^\ast =\set{k+n\Z}{\ggT(k,n)=1}
		\end{equation*}

		\item $|(\Z/n\Z)^\ast|=\varphi(n)$ und $\varphi(p)=p-1$ und falls $\ggT(a,b)=1$, dann $ \varphi(a*b)=\varphi(a)*\varphi(b)$ und $\varphi(n)=n*\prod_{p|n}(1-\frac1p)$

		\item $\Z/n\Z$ ist ein Körper genau dann, wenn $n$ eine Primzahl ist.

		\item Die lineare Abbildung $x\mapsto kx$ auf $\Z/n\Z$ ist genau dann bijektiv, wenn $\ggT(k,n)=1$.
		
		\item Sind $m,n$ teilerfremd, d.h. $\ggT(m,n)=1$, dann ist
		\begin{equation*}
			\pi:\Z\rightarrow\Z/m\Z\times\Z/n\Z, x\mapsto(x+m\Z,x+n\Z)
		\end{equation*}
		surjektiv. Damit erhält man eine bijektive Abbildung
		\begin{equation*}
			(x\mod mn)\mapsto (x\mod m,x\mod n)
		\end{equation*}
		\item \textbf{Chinesischer Restsatz}
		Für teilerfremde Zahlen $m,n$ ist die Abbildung
		\begin{equation*}
			\Z/mn\Z\rightarrow \Z/m\Z\times \Z/n\Z, 
			x+mn\Z\mapsto (x+m\Z,x+n\Z)
		\end{equation*}
		ein Isomorphismus (bijektiver Homomorphismus).

		Lösung des Kongruenzsystems
		\begin{align*}
			x\equiv a_1\mod m_1\\
			x\equiv a_2\mod m_2
		\end{align*}
		finden. Bestimme die Inversen $x_i$
		\begin{align*}
			m_2*x_1\equiv 1 \mod m_1\\
			m_1*x_2\equiv 1 \mod m_2
		\end{align*}
		Dann ist die Lösung
		\begin{equation*}
			x\in (x_1m_2a_1+x_2m_1a_2)+m_1m_2\Z
		\end{equation*}

		Es gibt eine Lösung genau dann, wenn für alle Paare $i\neq j$ gilt
		\begin{equation*}
			a_i\equiv a_j\mod \ggT(m_i,m_j)
		\end{equation*}

		\item \textbf{Der kleine Satz von Fermat} Für alle Primzahlen $p$ und alle $a\in\Z$ gilt
		\begin{equation*}
			a^p\equiv a\mod p
		\end{equation*}
		Falls $a$ und $p$ teilerfremd sind, gilt sogar
		\begin{equation*}
			a^{p-1}\equiv 1\mod p
		\end{equation*}

		\item \textbf{Satz von Euler} Für teilerfremde ganze Zahlen $a,n$ gilt
		\begin{equation*}
			a^{\varphi(n)}\equiv 1 \mod n
		\end{equation*}
		Verallgemeinert man den Satz (die multiplikative Gruppe $(\Z/n\Z)^\ast$ ist kommutativ), erhält man den \textbf{Satz von Lagrange} Es gilt sogar für jede kommutative Gruppe $G$ und jedes $a\in G$ 
		\begin{equation*}
		 	a^{|G|}=1
		\end{equation*} 

		\item Summe über die $\varphi(t)$ der Teiler von $n$ ist gleich $n$
		\begin{equation*}
			\sum_{t|n}\varphi(t)=n
		\end{equation*}

		\item \textbf{Satz von Wilson} Für alle natürlichen Zahlen $n\geq 2$ gilt
		\begin{equation*}
			(n-1)!\equiv -1\mod n\quad\Leftrightarrow\quad n \text{ ist Primzahl}
		\end{equation*}
	\end{itemize}


	\section{Graphen}
	$P_n$ ist der Pfad, $C_n$ ist der Kreis, $K_n$ ist der vollständige Graph, $K_{m,n}$ ist der vollständige bipartite Graph. Stier ist $\forall$ mit 5 Knoten.

	\begin{itemize}
		\item Die Summe aller Knotengrade in einem ungerichteten Graphen ist immer gerade.
		\item In jedem endlichen Graph ist die Anzahl der Knoten mit ungeradem Grad gerade.
		\item Ein zusammenhängender endlicher Graph hat genau dann einen Eulerpfad, wenn die Anzahl der Knoten mit ungeradem Grad maximal 2 ist. Ein Eulerkreis existiert genau dann, wenn alle Knoten geraden Grad haben.
		\item \textbf{Eulerformel} In endlichen zusammenhängenden planaren Graphen mit $n\geq 1$ Knoten, $m$ Kanten und $f$ Facetten gilt
		\begin{equation*}
			n-m+f=2\text{\quad bezeihungsweise\quad}n-m+f=z+1
		\end{equation*}
		für $z$ Zusammenhangskomponenten.
		Wichtige Folgerungen hieraus sind
		\begin{itemize}
			\item Ein planarer Graph mit $n\geq 3$ Knoten hat höchstens $3n-6$ Kanten.
			\item Ein planarer bipartiter Graph mit $n\geq 4$ Knoten hat höchstens $2n-4$ Kanten
			\item In jedem planaren Graph gibt es mindestens einen Knoten mit Grad kleiner oder gleich $5$.
			\item Der $K_5$ und der $K_{3,3}$ sind nicht planar.
		\end{itemize} (Folie 25.4)

		\item \textbf{Satz von Kuratowski} Ein Graph ist genau dann planar, wenn er keine Unterteilung des $K_5$ oder des $K_{3,3}$ enthält.

		\item \textbf{Satz von Ramsey} Fur alle $n\in N$ gibt es ein $N\in\N$, so dass jeder Graph mit $N$ Knoten eine Clique der Größe $n$ oder eine unabhängige Menge der Größe $n$ enthält.
	\end{itemize}



	\section{Zahlen und Abschätzungen}
	\begin{itemize}
		\item \textbf{Fibonacci-Zahlen} $F_0=0, F_1=1$. 
		\begin{equation*}
			F_{n}\leq 2^n\leq F_{2n}\quad \forall n\geq 3
		\end{equation*}
		\begin{equation*}
			F_n=\frac{1}{\sqrt 5}\left(\left(\frac{1+\sqrt 5}{2}\right)^n-\left(\frac{1-\sqrt 5}{2}\right)^n\right)
		\end{equation*}

		\begin{equation*}
			\ggT(F_n,F_m)=F_{\ggT(m,n)}
		\end{equation*}

		\item \textbf{Catalanzahlen}
		\begin{equation*}
			C_n=\frac{1}{n+1}\binom{2n}{n}=\frac{1}{2n+1}\binom{2n+1}{n}\sim\frac{4^n}{n\sqrt{\pi n}}
		\end{equation*}

		\textbf{Dyckwörter} $D_n$ ist die Menge der Dyckwörter mit $n$ Klammerpaaren, d.h. Länge $2n$. $|D_n|=C_n$.

		\textbf{Saturierte Binärbäume} Binärbäume, dessen Knoten entweder zwei oder keine Nachfolger haben. Zuordnung von Saturierten Bäumen zu normalen Binärbäumen ist bijektiv durch Weglassen aller Blätter. Es gibt $C_n$ viele saturierte Binärbäume mit $n$ inneren Knoten.

		\item \textbf{Abschätzungen von Laufzeiten}
		\begin{align*}
			n^n&\in \omega(n^c n!)\\
			\log(n!)&\in\Theta (n\log n)\\
			\log(\sqrt n)&\in \Theta(\log(\sqrt n))
		\end{align*}

	\end{itemize}

	\section{Sonstiges}
	\begin{itemize}
		\item \textbf{Markov-Ungleichung}, Sei $\forall \omega: X(\omega)\geq 0$ und $E(X)>0$, dann
		\begin{equation*}
			\forall \lambda >0: \enspace P(X\geq \lambda E(X))\leq \frac 1\lambda
		\end{equation*}

		\item $Var(X)=E(X^2)-E(X)^2=E((X-E(X))^2)$.
		\item $Var(a*X+b)=a^2*X, Var(X+Y)=Var(X)+Var(Y)$

		\item \textbf{Kombinatorik} Zurücklegen / Reihenfolge
		\begin{itemize}
			\item Ja/Ja: $n^k$
			\item Nein/Ja: $k!\binom nk$
			\item Ja/Nein: $\binom{n+k-1}{k}$
			\item Nein/Nein: $\binom nk$
		\end{itemize}

		\textbf{Partitionszahlen}
		Auf wieviele Arten kann man $n$ als Summe von $k\in \N$ schreiben? Diese Anzahl nennen wir $P(n,k)$. Z.B. Aufteilung einer $n$-elementigen Menge in $k$ nichtleere Teilmengen.
		\begin{equation*}
			P(n,k)=P(n-1,k-1)+P(n-k,k)=\sum_{j\leq k}P(n-k,j)
		\end{equation*}
	\end{itemize}
\end{multicols}
\end{document}
