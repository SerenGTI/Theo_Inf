\chapter{Deterministisch Kontextfreie Sprachen $\dcfl$:}
\begin{equation*}
	\dcfl\subset\cfl\subset\csl\subset\rec\subset\re
\end{equation*}
Wort-, Leerheits- und Äquivalenzproblem entscheidbar. Abschluss nur unter Komplement.

Beschreib- und erkennbar durch einen endlichen deterministischen Kellerautomaten (DPDA).



\section{Automatenmodell DPDA:}\label{dcfl:dpda}
Der deterministische Kellerautomat ist ähnlich definiert wie ein der nichtdeterministische (Siehe \autoref{cfl:pda}).

Der Unterschied zum PDA liegt dabei, dass beim DPDA in jeder Situation nur ein Übergang möglich sein darf,
\begin{equation*}
	\forall z\in Z, a\in\Sigma, A\in\Gamma : |\delta(z,a,A)|+|\delta(z,\epsilon,A)|\leq 1
\end{equation*}
und der DPDA akzeptiert nicht durch leeren Keller sondern durch Endzustände.

\paragraph{Ein deterministischer Kellerautomat ist ein 7-Tupel}
\begin{equation*}
	M=(Z,\Sigma,\Gamma,\delta,z_0,\#, E)
\end{equation*}
\begin{description}
	\item[$Z$] endliche Zustandsmenge
	\item[$\Sigma$] Eingabealphabet
	\item[$\Gamma$] Kelleralphabet
	\item[$\delta$] Überführungsfunktion $\delta:Z\times(\Sigma\cup\simpleset\epsilon)\times\Gamma \rightarrow Z\times\Gamma^\star$
	\item[$z_0$] Startzustand, $z_0\in Z$
	\item[$\#$] Keller-Bottom-Symbol $\#\in \Gamma\setminus\simpleset\Sigma$
	\item[$E$] Endzustandsmenge $E\subseteq Z$
\end{description}

\paragraph{Akzeptierte Sprache eines deterministischen PDA:}
\begin{equation*}
	N(M)\coloneqq\set{w\in \Sigma^\star}{\exists e\in E, V\in \Gamma^\star: (z_0, w, \#)\vdash^\star (e,\epsilon,V)}
\end{equation*}
dies wird auch als \emph{Akzeptieren durch Endzustand} bezeichnet.

Beide Akzeptanzarten, (durch Endzustand und leeren Keller, siehe \autoref{cfl:pda}) sind äquivalent.


\section{Sätze zu den deterministisch kontextfreien Sprachen:}
\begin{itemize}
	\item Eine deterministisch kontextfreie Sprache geschnitten mit einer regulären Sprache ist wieder eine deterministisch kontextfreie Sprache.
	\begin{equation*}
		L_1\in\dcfl, L_2\in\reg \Rightarrow L_1\cap L_2\in\dcfl
\end{equation*}
\end{itemize}

\section{Beispiele: }
\begin{itemize}
	\item $L_1=\set{w\$w^R}{w\in\Sigma^\star}$ (markierte Palindrome)
\end{itemize}
