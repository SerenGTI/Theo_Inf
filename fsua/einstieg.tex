\chapter{Einstieg}
\begin{definition}{Alphabet}
	Als Alphabet bezeichnen wir eine endliche, nichtleere Menge, deren Elemente Buchstaben genannt werden.

	Dieses wird üblich mit $\Sigma$ bezeichnet.
\end{definition}

\begin{definition}{Freies Monoid über $\Sigma$}
	Ein Monoid ist eine Menge mit einer assoziativen Verknüpfung und einem neutralen Element. Die Menge aller endlichen Zeichenketten, die sich aus Elementen von $\Sigma$ bilden lassen bilden mit der \emph{Konkatenation} ein Monoid $\Sigma^\star$.

	Das leere Wort ($\epsilon$) bildet das neutrale Element.
\end{definition}


\begin{definition}{Grammatik}
	Eine Grammatik ist ein Quadrupel:
	\begin{equation*}
		G=(V,\Sigma, P, S)
	\end{equation*}
	\begin{description}
		\item[$V$] Eine Menge an Zeichen, den Variablen
		\item[$\Sigma$] Das Alphabet, $V\cap\Sigma =\emptyset$
		\item[$P$] Die Produktionsmenge, $P\subseteq(V\cup\Sigma)^+\times(V\cup\Sigma)^\star$
		\item[$S$] Das Startsymbol oder die Startvariable, $S\in V$
	\end{description}
\end{definition}
