\chapter{Kontextfreie Sprachen}
\begin{equation*}
	\text{Typ-2} = \cfl\subset\csl\subset\rec\subset\re
\end{equation*}
\section{Automatenmodell PDA}\label{cfl:pda}
Der Push DownAutomat (Kellerautomat) ist ähnlich definiert wie ein nichtdeterministischer endlicher Automat (Siehe \autoref{reg:nea}).

Der PDA jedoch hat einen entscheidenden Unterschied, er hat einen sogenannten \emph{Kellerspeicher}, in dem Informationen zwischengespeichert werden können.

\paragraph{Ein Kellerautomat ist ein 6-Tupel}
\begin{equation*}
	M=(Z,\Sigma,\Gamma,\delta,z_0,\#)
\end{equation*}
\begin{description}
	\item[$Z$] endliche Zustandsmenge
	\item[$\Sigma$] Eingabealphabet
	\item[$\Gamma$] Kelleralphabet
	\item[$\delta$] Überführungsfunktion $\delta:Z\times(\Sigma\cup\simpleset\epsilon)\times\Gamma\rightarrow \Pot_e(Z\times\Gamma^\star)$
	\item[$z_0$] Startzustand, $z_0\in Z$
	\item[$\#$] Keller-Bottom-Symbol $\#\in \Gamma\setminus\simpleset\Sigma$
\end{description}

\paragraph{Akzeptierte Sprache eines nichtdeterministischen PDA}
\begin{equation*}
	N(M)\coloneqq\set{w\in \Sigma^\star}{\exists z\in Z: (z_0, w, \#)\vdash^\star (z,\epsilon,\epsilon)}
\end{equation*}
dies wird auch als \emph{Akzeptieren durch leeren Keller} bezeichnet.

Insbesondere beim deterministischen Kellerautomaten gibt es eine zweite Definition der akzeptierten Sprache, beide Definitionen sind äquivalent (Siehe \autoref{dcfl:dpda}).

\paragraph{Konfiguration des PDA}
Eine Konfiguration ist jedes Element $k$ aus der Menge $Z\times\Sigma^\star\times\Gamma^\star$

Einen Konfigurationsübergang stellt man durch $\vdash$ dar.

Die aktuelle Konfiguration sei $(z,a_1a_2\ldots a_n,A_1\ldots A_m)$, möglich sind jetzt die Übergänge aus $\delta(z,a_1,A_1)$ und $\delta(z,\epsilon,A_1)$.

Ein möglicher Übergang ist also für $(z',B_1\ldots B_k)\in\delta(z,a_1,A_1)$:
\begin{equation*}
	(z,a_1a_2\ldots a_n,A_1\ldots A_m)\vdash (z',a_2\ldots a_n,B_1\ldots B_kA_2\ldots A_m)
\end{equation*}



\section{Sätze zu $\cfl$}
\begin{itemize}
	\item Alle Typ-2 Sprachen über einem einelementigen Alphabet sind bereits regulär.
	\item Die Kontextfreien Sprachen sind gegen Substitution abgeschlossen.
	\item Die Klasse der kontextfreien Sprachen ist abgeschlossen unter Sternoperation, Vereinigung und Konkatenation.

	Zur Vereinigung: $G_1=(V_1,\Sigma, P_1,S_1), G_2=(V_2,\Sigma, P_2,S_2), V_1\cap V_2=\emptyset, S\not\in V_1\cup V_2$.
	\begin{equation*}
		G=(V_1\cup V_2, \Sigma, P_1\cup P_2\cup\simpleset{(S,S_1),(S,S_2)},S) \ \rightsquigarrow\  L(G)=L(G_1)\cup L(G_2)
	\end{equation*}
	\item Das Wortproblem ($\mathcal O(n^3)$) sowie das Leerheits- und Endlichkeitsproblem ist entscheidbar (Siehe \autoref{algo:cyk}).
	\item Die Klasse der Typ-2 Sprachen ist identisch zu der durch EBNF beschreibbaren Sprachen.
\end{itemize}
\subsection{Pumping-Lemma für Typ-2}
Sei $L\subseteq \Sigma^\star$ eine kontextfreie Sprache, dann gibt es eine Zahl $n$ so, dass für alle $z\in L$ mit $|z|\geq n$ eine Zerlegung mit $z=uvwxy$ in $u,v,w,x,y\in\Sigma^\star$ exisitert für die die drei Bedingungen erfüllt sind:
\begin{itemize}
	\item $|vx|\geq 1$
	\item $|vwx|\leq n$
	\item $\forall i\in\N: uv^iwx^iy\in L$
\end{itemize}


\section{Chomsky-Normalform}
Eine Typ-2 Grammatik $(V,\Sigma,P,S)$ ist in Chomsky-Normalform (CNF),  wenn gilt:
\begin{equation*}
	\forall (u,v)\in P: v\in V^2\cup \Sigma
\end{equation*}
Zu jeder Typ-2 Grammatik existiert eine Grammatik $G'$ in CNF, für die gilt $L(G)=L(G')$!

Für alle Ableitungen in CNF gilt: Die Ableitung eines Wortes der Länge $n$ benötigt genau $2n-1$ Schritte!
\subsection{Umformungsalgorithmus}\label{cnf:algorithmus}
\begin{enumerate}
	\item Zunächst wollen wir erreichen, dass folgendes gilt: $(u,v)\in P\Rightarrow (|v|>1 \vee v\in \Sigma)$
	\begin{enumerate}
		\item \textbf{Ringableitungen entfernen:}

		Eine Ringableitung liegt vor, wenn es Variablen $V_1,\ldots V_r$ gibt, die sich im Kreis in einander ableiten lassen, d.h. es gibt Regeln $V_i\rightarrow V_{i+1}$ und $V_r\rightarrow V_1$.

		Um dies loszuwerden, werden alle Variablen $V_i$ durch eine neue Variable $V$ ersetzt. Überflüssige Regeln wie $V\rightarrow V$ werden gelöscht.
		\item \textbf{Variablen anordnen:}

		Man legt eine Ordnung der Variablen fest: $V=\simpleset{A_1, A_2, \ldots, A_n}$, hierfür muss gelten:
		\begin{equation*}
			A_i\rightarrow A_j \in P \Leftrightarrow i<j
		\end{equation*}
		Falls dies nicht gilt, müssen Abkürzungen verwendet werden, also alle Produktionen von $A_j$ werden eingesetzt:
		\begin{equation*}
			P=(P\setminus\simpleset{A_i\rightarrow A_j})\cup\set{(A_i,w)}{(A_j,w)\in P}
		\end{equation*}
	\end{enumerate}
	\item Jetzt gilt für jede Regel $(u,v)\in P$ entweder $v\in\Sigma$ oder $|v|\geq 2$.

	Für letztere Regeln werden nun \textbf{Pseudoterminale} eingeführt. Es werden neue Variablen und Produktionen für jedes Terminalsymbol hinzugefügt, z.B. $V_a\rightarrow a$.

	\item \textbf{Letzter Schritt:} Alle rechten Seiten mit $|v|>2$ müssen nun noch auf Länge 2 gekürzt werden. Hierfür werden wiederum neue Variablen eingefügt:%
	\begin{align*}
		A\rightarrow C_1C_2C_3\\
		\intertext{Wird gekürzt zu}
		A\rightarrow C_1D_1\\
		D_1\rightarrow C_2C_3
	\end{align*}
\end{enumerate}

\section{Greibach-Normalform}
Eine Typ-2 Grammatik $(V,\Sigma,P,S)$ ist in Greibach-Normalform (GNF),  wenn gilt:
\begin{equation*}
	\forall (u,v)\in P: v\in \Sigma V^\star
\end{equation*}
Zu jeder Typ-2 Grammatik existiert eine Grammatik $G'$ in GNF, für die gilt $L(G)=L(G')$!

\subsection{Umformungsalgorithmus}
\begin{enumerate}
	\item \textbf{Der erste Algorithmus:}

	Der erste Algorithmus hat zum Ziel, dass alle Produktionen einer bestimmten Ordnung unterliegen.
	Hierfür werden zunächst die Variablen angeordnet:
	\begin{equation*}
		V=\simpleset{A_1,A_2,\ldots,A_m}
	\end{equation*}
	Nun sollen nur Regeln $A_i\rightarrow A_j\alpha$ in $P$ sein, wenn $i<j$ gilt.

	Um dies zu erreichen wird solange Eingesetzt bis keine Regeln dieser Form vorliegen.

	Für alle $A_i\rightarrow A_j\alpha \in P$ mit $i> j$ werden alle Produktionsregeln
	\begin{equation*}
		A_j\rightarrow \beta_1|\ldots|\beta_r
	\end{equation*}
	Eingesetzt zu
	\begin{equation*}
		A_i\rightarrow \beta_1\alpha|\ldots|\beta_r\alpha
	\end{equation*}
	Und schließlich die Regel $A_i\rightarrow A_j\alpha$ aus $P$ gestrichen.

	\item \textbf{Beseitigung von Linksrekursion:}

	Alle Produktionsregeln sind von der Form:
	\begin{align*}
		A&\rightarrow A\alpha_1|\ldots|A\alpha_k|\beta_1|\ldots|\beta_l\\
		\intertext{Diese können durch diese $2k+2l$ Regeln ersetzt werden:}
		A&\rightarrow \beta_1|\ldots|\beta_l\\
		A&\rightarrow \beta_1B|\ldots|\beta_lB\\
		B&\rightarrow \alpha_1|\ldots|\alpha_k\\
		B&\rightarrow \alpha_1B|\ldots|\alpha_kB
	\end{align*}%
	Nun sind keinerlei Linksrekursionen mehr vorhanden!

	\item \textbf{Der zweite Algorithmus:}

	Der zweite Algorithmus hat zum Ziel, dass alle rechten Seiten mit einem Terminalsymbol beginnen.

	Da die Regeln mit $A_m$ auf der linken Seite nur in ein Terminal übergehen können, da sie am Ende der Variablenanordnung stehen müssen wir nur alle $A_m$-Produktionen bei den $A_{m-1}$-Regeln einsetzen und so weiter.

	\item \textbf{Der letzte Schritt:}

	Als letztes müssen wir überprüfen ob die $B$-Regeln aus der Beseitigung der Linksrekursionen die gewünschte Form haben.
	Da aber alle $B$-Produktionen entweder mit einem $A_i$ oder einem Terminal beginnen, muss wieder nur eingesetzt werden.

	Schließlich müssen die Terminalsymbole, die in allen Produktionen weiter hinten auftreten durch Pseudoterminale ersetzt werden.
\end{enumerate}

\section{Beispiele}
\begin{itemize}
	\item $L_1=\set{ww^R}{w\in\Sigma^\star}$ (unmarkierte Palindrome)
	\item Korrekt geklammerte arithmetische Ausdrücke (Dyck-Wörter)
\end{itemize}
