\chapter{Überblick}
Ein kurzer Gesamtüberblick über den Zusammenhang von formalen Sprachen zur Komplexität und Berechenbarkeitstheorie:

\definecolor{entscheidbar}{rgb}{0.4, 0.9, 0.7}
	\begin{center}
		\begin{tikzpicture}
			\tikzset{
			grouplabel/.style={
			draw,
			fill = white,
			rectangle,
			inner sep = 4pt,
			rounded corners=1pt
			}
			}

			\draw [rounded corners=5pt, dotted, line width=0.2mm] (0,0) rectangle (14,16.5);
			\draw node at (7,16.5) [grouplabel] {alle formalen Sprachen};


			% CO-SEMI-ENTSCHEIDBAR
			\draw [rounded corners=5pt, dotted, line width=0.5mm] (0.5,0.5) rectangle (12,12);%co-semi
			\draw node at (6,0.5) [grouplabel] {co-semi-entscheidbare Sprachen};
			\draw node at (0.5,1.3) [label={right:$\chi'_{\overline L}(w)$ Turing-berechenbar, Komplement ist semi-entscheidbar}] {};

			% SEMI-ENTSCHEIDBAR
			\draw [rounded corners=5pt, dotted, line width=0.5mm] (2,2) rectangle (13.5,15.5);%semi
			\draw node at (8,15.5) [grouplabel] {\hyperref[sec:typ0]{Typ-0$=\re$, semi-entscheidbare Sprachen}};
			\draw node at (2,14.8) [label={right:$\operatorname{PCP, K, H, H_0, \ldots}$}] {};
			\draw node at (2,14.3) [label={right:$\chi'_{L}(w)$ Turing-berechenbar}] {};
			\draw node at (2,13.7) [label={right:$L=T(M) \rightsquigarrow$ wird durch eine TM akzeptiert}] {};
			\draw node at (2,13.2) [label={right:$f:\N\rightarrow \Sigma^\star, f(\N)=L \rightsquigarrow$ rekursiv aufzählbar}] {};
			\draw node at (2,12.7) [label={right:$f:\Sigma^\star\rightarrow\Sigma^\star, f(\Sigma^\star)=L \rightsquigarrow$ Wertebereich einer berechenbaren Funktion}] {};




			%ENTSCHEIDBAR
			\draw [draw, fill=accent!50, line width=0.5mm] (2,2) rectangle (12,12);%entscheidbar
			\draw node at (7,12) [grouplabel] {\hyperref[sec:rec]{entscheidbare Sprachen}};
			\draw node at (2,10.5) [label={right:$\chi_{L}(w)$ Turing-berechenbar}] {};



			%% TYP1
			\draw [rounded corners=2pt, line width=0.2mm] (2.2,2.2) rectangle (11.8,9.9);
			\draw node at (7,9.9) [grouplabel] {\hyperref[sec:typ1]{Typ-1$=\csl$}};
			\draw node at (2.2,9.2) [label={right:$a^nb^nc^n, a^{2^n}, \ldots$}] {};
			\draw node at (2.2,8.7) [label={right:LBA}] {};

			%% TYP2
			\draw [rounded corners=2pt, line width=0.2mm] (2.4,2.4) rectangle (11.6,8.1);
			\draw node at (7,8.1) [grouplabel] {\hyperref[sec:typ2]{Typ-2$=\cfl$}};
			\draw node at (2.4,7.4) [label={right:$a^na^n, ww^R, \ldots$}] {};
			\draw node at (2.4,6.9) [label={right:PDA}] {};

			% DCFL
			\draw [rounded corners=2pt, line width=0.2mm] (2.6,2.6) rectangle (11.4,6.3);%dcfl
			\draw node at (7,6.3) [grouplabel] {\hyperref[sec:typ2]{$\dcfl$}};
			\draw node at (2.6,5.6) [label={right:$a^nb^n, w\$w^R, \ldots$}] {};
			\draw node at (2.6,5.1) [label={right:DPDA}] {};

			% TYP3
			\draw [rounded corners=2pt, line width=0.2mm] (2.8,2.8) rectangle (11.2,4.5);
			\draw node at (7,4.5) [grouplabel] {\hyperref[sec:typ3]{Typ-3$=\reg$}};
			\draw node at (2.8,3.8) [label={right:$\Sigma^\star, \emptyset, a^\star,\ldots$}] {};
			\draw node at (2.8,3.3) [label={right:DEA, NEA, reguläre Ausdrücke}] {};

		\end{tikzpicture}
	\end{center}
