\chapter{Reguläre Sprachen $\reg$: Typ-3}
\begin{equation*}
	\reg\subset\dcfl\subset\cfl\subset\csl\subset\rec\subset\re
\end{equation*}

\section{Automatenmodell DEA:}
Ein deterministischer endlicher Automat ist ein 5-Tupel
\begin{equation*}
	M=(Z,\Sigma,\delta,z_0,E)
\end{equation*}
\begin{description}
	\item[$Z$] endliche Zustandsmenge
	\item[$\Sigma$] Eingabealphabet
	\item[$\delta$] Überführungsfunktion $\delta:Z\times\Sigma\rightarrow Z$
	\item[$z_0$] Startzustand, $z_0\in Z$
	\item[$E$] Endzustandsmenge, $E\subseteq Z$
\end{description}
\bigskip
Es lässt sich außerdem eine erweiterte Funktion $\hat\delta$ definieren:
\begin{align*}
	&\hat\delta:Z\times\Sigma^\star\rightarrow Z\\
	\intertext{Mit den folgenden Eigenschaften:}
	&\hat\delta(z,\epsilon)=z\\
	&\hat\delta(z,ax)=\hat\delta(\delta(z,a),x)
\end{align*}
Die von einem deterministischen Automaten $M$ akzeptierte Sprache ist
\begin{equation*}
	T(M)=\set{w\in\Sigma^\star}{\hat\delta(z_0,w)\in E}
\end{equation*}

\section{Automatenmodell NEA:}\label{reg:nea}
Ein nichtdeterministischer endlicher Automat ist ein 5-Tupel
\begin{equation*}
	M=(Z,\Sigma,\delta,S,E)
\end{equation*}
\begin{description}
	\item[$Z$] endliche Zustandsmenge
	\item[$\Sigma$] Eingabealphabet
	\item[$\delta$] Überführungsfunktion $\delta:Z\times\Sigma\rightarrow \Pot(Z)$
	\item[$S$] Startzustandsmenge, $S\subseteq Z$
	\item[$E$] Endzustandsmenge, $E\subseteq Z$
\end{description}
Der NEA ist formal stärker als der DEA, sie akzeptieren jedoch beide die selbe Sprachklasse.

Die akzeptierte Sprache eines nichtdeterministischen endlichen Automaten ist:
\begin{equation*}
	T(M)=\set{w\in\Sigma^\star}{\hat\delta(S,w)\cap E\neq \emptyset}
\end{equation*}

\section{Reguläre Ausdrücke}
Die regulären Sprachen lassen sich zusätzlich zu den zwei Automatenmodellen auch durch sog. reguläre Ausdrücke beschreiben. Eine Definition für die Syntax der regulären Ausdrücke ist:
\begin{itemize}
	\item $\emptyset$ und $\epsilon$ sind reguläre Ausdrücke.
	\item $a$ ist ein regulärer Ausdruck für alle $a\in\Sigma$
	\item Wenn $\alpha$ und $\beta$ eguläre Ausdrücke sind, dann sind auch $\alpha\beta$, $(\alpha|\beta)$ und $(\alpha)^\star$ reguläre Ausdrücke.
\end{itemize}
Die Semantik der regulären Ausdrücke ist ebenso induktiv bestimmt:
\begin{itemize}
	\item $L(\emptyset)=\emptyset$ und $L(\epsilon)=\simpleset{\epsilon}$
	\item $L(a)=\simpleset{a}$ für jedes $a\in\Sigma$
	\item $L(\alpha \beta)=L(\alpha)L(\beta)$, $L(\alpha|\beta)=L(\alpha)\cup L(\beta)$, $L((\alpha)^\star)=L(\alpha)^\star$
\end{itemize}






\section{Sätze zu den regulären Sprachen:}
\begin{itemize}
	\item Typ-3 Sprachen können \emph{nicht} inhärent mehrdeutig sein, da sich zu jeder Sprache ein Minimalautomat bilden lässt.
	\item Die Klasse der Typ-3 Sprachen ist unter allen boole'schen Operationen, Sternoperation und der Konkatenation abgeschlossen.
	\item Für reguläre Sprachen ist das Wortproblem (in Linearzeit), das Leerheitsproblem, das Äquivalenzproblem sowie das Schnittproblem entscheidbar.
	\item Alle Typ-2 Sprachen über einem einelementigen Alphabet sind bereits regulär.
\end{itemize}
\subsection{Pumping-Lemma für Typ-3}%
	Für jede reguläre Sprache $L$ gibt es ein $n\in\N$, so dass für jedes $x\in L$ mit $|x|\geq n$ eine Zerlegung in drei Teile exisitert: $x=uvw$, so dass die drei Bedingungen erfüllt sind:
	\begin{itemize}
		\item $|v|\geq 1$
		\item $|uv|\leq n$
		\item $\forall i \in\N: uv^iw\in L$ (\glqq Pump-Bedingung\grqq)
	\end{itemize}
	\begin{align*}
		\intertext{Gilt die Negation dieser Aussage, also}
		\forall n\in\N:\exists x\in L, |x|\geq n : \forall u,v,w \in\Sigma^\star, x=uvw, |v|\geq 1, |uv|\leq n:\exists i\in\N:uv^iw\not\in L
		\intertext{so ist L nicht regulär!}
 	\end{align*}%

	\textbf{ABER:} Das Pumping-Lemma gibt keine Charakterisierung der Typ-3 Sprachen an! Es gibt auch Sprachen, die nicht Typ-3 sind, die Aussage des Lemmas aber trotzdem erfüllen!

	Das Pumping-Lemma gibt also nur eine Möglichkeit, zu Beweisen, dass eine Sprache \emph{nicht} regulär ist! (Siehe \autoref{bsp:pumpingLemma})
\subsection{Myhill-Nerode-Äquivalenz}%
	Mit der Myhill-Nerode-Äquivalenz ist es möglich nachzuweisen, ob eine Sprache regulär ist.
	\begin{align*}
		&x \mathrm R_L y \Longleftrightarrow \left[\forall w\in\Sigma^\star : xw\in L \Leftrightarrow yw\in L\right]\\
		\intertext{Bzw. anhand eines DEA (dies führt zu einer Verfeinerung von $\mathrm R_L$)}
		&x \mathrm R_M y \Longleftrightarrow \left[\delta (z_0,x)=\delta (z_0,y)\right]\\
		\intertext{Es gilt:}
		&x \mathrm R_M y \Rightarrow \forall w\in\Sigma^\star : \delta (z_0,xw)=\delta (z_0,yw) \Rightarrow x \mathrm R_L y
	\end{align*}
	Die Sprache $L\subseteq \Sigma^\star$ ist genau dann regulär, wenn der Index der Myhill-Nerode-Äquivalenz $\mathrm R_L$endlich ist.
\subsection{Erkennung durch Monoide -- Syntaktisches Monoid}
Sei $L\subseteq \Sigma^\star$ eine formale Sprache und $M$ ein Monoid.

$M$ erkennt $L$, wenn eine Teilmenge $A\subseteq M$ und ein Homomorphismus $\varphi:\Sigma^\star \rightarrow M$ existiert, so dass gilt:
\begin{align*}
	L&=\varphi^{-1}(A) &&\text{d.h. }w\in L \Leftrightarrow \varphi(w)\in A\\
	L&=\varphi^{-1}(\varphi(L)) &&\text{d.h. }w\in L \Leftrightarrow \varphi(w)\in \varphi(L)
\end{align*}

Weiter kann man für eine konkrete Sprache $L$ die \emph{syntaktische Kongruenz} definieren:
\begin{align*}
	w_1\equiv_Lw_2 \Longleftrightarrow [\forall x,y\in\Sigma^\star : xw_1y\in L \Leftrightarrow xw_2y\in L]
\end{align*}

Basierend auf dieser Kongruenz definieren wir das Quotientenmonoid der Kongruenz, dessen Elemente die Äquivalenzklassen sind.

Das Quotientenmonoid oder auch syntaktisches Monoid bezüglich der syntaktischen Kongruenz wird mit
\begin{equation*}
	\mathrm{Synt}(L)\coloneqq\left(\Sigma^\star/\equiv_L\right)
\end{equation*}
bezeichnet.

Für jede Sprache $L$ gibt es ein syntaktisches Monoid das $L$ mit dem Homomorphismus
\begin{equation*}
	\varphi: L\rightarrow \mathrm{Synt}(L), w\mapsto [w]
\end{equation*}
erkennt.
Ist $|\mathrm{Synt}(L)|$ endlich, so ist $L$ regulär bzw. erkennbar.



\section{Beispiele: }
\begin{itemize}
	\item $L_1=\Sigma^\star$
	\item $L_2=L(a(a|b)^\star bb(ab)^\star)$
	\item Automat $M$ mit $L_3=T(M)=\set{(ab)^n}{n\in\N_0}$:\\

	\vspace{1em}
	\begin{tikzpicture}[->,>=stealth',shorten >=1pt,auto,node distance=2cm,
	                    semithick]
	  \tikzstyle{every state}=[fill=none,draw=black,text=black]

	  \node[initial,state,accepting] (A)                    {$q_a$};
	  \node[state]         (B) [right of=A] {$q_b$};
	  \node[state]         (C) [below of=A] {$q_f$};

	  \path (A) edge [bend left]  node {a} (B)
	            edge              node {b} (C)
	        (B) edge [bend left]  node {b} (A)
	            edge [bend left]  node {a} (C)
	        (C) edge [loop left]  node {a,b} (C);
	\end{tikzpicture}
\end{itemize}
