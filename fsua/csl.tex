\chapter{Kontextsensitive Sprachen}
\begin{equation*}
	\text{Typ-1} = \csl\subset\rec\subset\re
\end{equation*}

\section{Automatenmodell Turingmaschine}
\paragraph{Eine Turingmaschine ist ein 7-Tupel}
\begin{equation*}
	M=(Z,\Sigma,\Gamma,\delta,z_0,\Box,E)
\end{equation*}
\begin{description}
	\item[$Z$] endliche Zustandsmenge
	\item[$\Sigma$] Eingabealphabet $\Sigma\subseteq\Gamma$
	\item[$\Gamma$] Bandalphabet
	\item[$\delta$] Überführungsfunktion $\delta: Z\times\Gamma\rightarrow Z\times\Gamma\times\simpleset{L,R,N}$
	\item[$z_0$] Startzustand, $z_0\in Z$
	\item[$\Box$] Leersymbol, $\Box\in\Gamma\setminus\Sigma$
	\item[$E$] Endzustandsmenge $E\subseteq Z$
\end{description}

\paragraph{Konfiguration}
Eine Konfiguration $k$ einer Turingmaschine definieren wir als ein Element der Menge
\begin{equation*}
	k\in\Gamma^\star Z\Gamma^\star
\end{equation*}
Das bedeutet, dass sich die Turingmaschine in der Konfiguration $k=\alpha z\beta$ im Zustand $z$ befindet und der Schreib- Lesekopf das erste Symbol von $\beta$ liest.


\paragraph{Die akzeptierte Sprache einer Turingmaschine}
Die akzeptierte Sprache einer Turingmaschine ist definiert als
\begin{equation*}
	T(M)\coloneqq\set{w\in\Sigma^\star}{\exists \alpha,\beta\in\Gamma^\star, e\in E: z_ow\vdash^\star \alpha e\beta}
\end{equation*}
Wenn $w\not\in T(M)$ können drei verschiedene Dinge geschehen:
\begin{enumerate}
	\item Die Turingmaschine hält an und ist nicht in einem Endzustand.
	\item Die Turingmaschine läuft weiter und kommt irgendwann in eine Schleife

	$z_ow\vdash^\star \alpha z\beta\vdash^\star\alpha'z'\beta'\vdash^\star\alpha z\beta\vdash\cdots$
	\item Die Turingmaschine rechnet endlos, ohne in eine Schleife zu kommen (Halteproblem).
\end{enumerate}

\subsection{Einschränkung LBA:}
Ein linear beschränkter Automat ist eine Turingmaschine, die bei der Verarbeitung der Eingabe niemals den Platz der Eingabe auf dem Arbeitsband verlässt.

Um das zu erreichen muss das letze Symbol der Eingabe markiert sein um den rechten Rand der Eingabe erkennbar zu machen, deshalb definieren wir:
\begin{equation*}
	\Sigma'=\Sigma\cup\set{\hat a}{a\in\Sigma}
\end{equation*}
Eine nichtdeterministische Turingmaschine nennen wir einen linear beschränkten Automaten, wenn gilt:
\begin{equation*}
	\forall a_1a_2\ldots a_{n-1}\hat a_n\in\Sigma^+, \alpha,\beta\in\Gamma^\star, z\in Z \text{ mit } z_0a_1a_2\ldots a_{n-1}\hat a_n\vdash^\star \alpha z \beta : |\alpha\beta|\leq n
\end{equation*}

Für die akzeptierte Sprache gilt dann:
\begin{equation*}
	T(M)\coloneqq\set{a_1\ldots a_n\in\Sigma^\star}{\exists \alpha,\beta\in\Gamma^\star, e\in E: z_oa_1\ldots\hat a_n\vdash^\star \alpha e\beta}
\end{equation*}

\textbf{Die Klasse der durch LBAs erkannten Sprachen ist gleich der Typ-1 Sprachen!}

Dabei ist die Frage ob nichtdeterministische und deterministische LBAs gleich mächtig sind noch offen.



\section{Sätze zu $\csl$}
\begin{itemize}
	\item Die Klasse der kontextsensitiven Sprachen ist abgeschlossen unter allen Operationen!
	\item Für die kontextsensitiven Sprachen und alle höheren Sprachklassen gibt es die sog. Epsilon-Sonderregel, die oft verwendet wird um trotz der Eigenschaft nichtverkürzend ein Wort der Länge Null zu erreichen:
	\begin{equation*}
		(S,\epsilon)\in P \ \text{ aber }\  \forall (u,v)\in P:S\not\in v
	\end{equation*}
\end{itemize}

\subsection{Algorithmus zur Entscheidbarkeit des Wortproblems:}
Die Funktion $\mathrm{Abl}_n(X)$ wird iteriert angewendet, bis sich entweder $X$ nicht mehr ändert ($w\not\in L(G)$) oder das gesuchte Wort $w$ in $X$ enthalten ist ($w\in L(G)$).

Dabei ist $n$ die Länge des gesuchten Worts $w$, also $|w|$.

Die Funktion $\mathrm{Abl}_n(X)$ ist für eine Grammatik $G$ wie folgt definiert:
\begin{equation*}
	\mathrm{Abl}_n(X)\coloneqq X\cup\set{w\in(V\cup\Sigma)^\star}{|w|\leq n \wedge \exists y\in X:y\Rightarrow_G w}
\end{equation*}


\section{Kuroda-Normalform}
Eine Typ-1 Grammatik $(V,\Sigma, P, S)$ ist in Kuroda-Normalform wenn alle Regeln von einem der vier Typen sind:
\begin{itemize}
	\item $A\rightarrow a$
	\item $A\rightarrow B$
	\item $A\rightarrow BC$
	\item $AB\rightarrow CD$
\end{itemize}
Wobei $A,B,C,D\in V$ und $a\in \Sigma$ ist.

Zu jeder Typ-1 Grammatik existiert eine äquivalente Grammatik in Kuroda-Normalform


\subsection{Umformungsalgorithmus}
\begin{enumerate}
	\item \textbf{Pseudoterminale:}
	Zunächst alle Terminalsymbole, die nicht alleine auf einer rechten Seite stehen durch Pseudoterminale ersetzen.

	Jetzt stören noch die Regeln $A_1\ldots A_n\rightarrow B_1\ldots B_m$ mit $1\leq n\leq m, m>2$

	\item Für Regeln der Form $A\rightarrow B_1\ldots B_m, m>2$ können wir die gleiche Methode wie bei den Typ-2 Normalformen anwenden (Siehe \autoref{cnf:algorithmus}).

	\item Regeln der Form $A_1\ldots A_n\rightarrow B_1\ldots B_m$ mit $2\leq n\leq m, m>2$ ersetzen wir durch:
	\begin{align*}
		&A_1D_1\rightarrow B_1D_1\\
		&D_1A_2A_1\ldots A_n\rightarrow B_2\ldots B_m
	\end{align*}
\end{enumerate}

\section{Beispiele}
\begin{itemize}
	\item $L_1=\set{a^nb^nc^n}{n\geq 1}$ (drei und mehr gleiche Exponenten)
	\item $L_2=\set{a^{n^2}}{n\in N^+}$
\end{itemize}
