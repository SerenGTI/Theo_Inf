\chapter{Anwedungen der Sätze}
\section{Pumping-Lemma für Typ-3}\label{bsp:pumpingLemma}
Für die Sprache $L=\set{a^nb^n}{n\geq 1}$:%
\begin{align*}
	\intertext{Zunächst wählt man ein Wort $x$, mit $|x|\geq n$:}
	x&=a^nb^n \in L,\quad |a^nb^n|=2n>n\\
	\intertext{Nun zu einer beliebigen Zerlegung $x=uvw$, für die die Bedingungen gelten:}
	x&=uvw=a^nb^n\\
	u&=a^{n-l-k} \quad v=a^l \quad w=a^k b^n \text{ mit $l\geq 1$}\\
	x&=(a^{n-l-k}) (a^l) (a^k b^n)\\
	\intertext{Pumpt man nun $v$ mit $v=0$:}
	x&=(a^{n-l-k}) (a^l)^i (a^k b^n)=(a^{n-l-k}) (a^l)^0 (a^k b^n)\\
	x&=(a^{n-l-k}) (a^k b^n) = a^{n-l} b^n \not\in L\text{, da $l\geq 1$}\\
\end{align*}


\section{Myhill-Nerode-Äquivalenz}\label{bsp:myhill}
Für die Sprache $L=\set{w\in\simpleset{a,b}^\star}{w \text{ enthält das Teilwort $abb$}}$:%
\begin{align*}
	\intertext{Finden der Äquivalenzklassen:}
	[\epsilon]&=\simpleset{\epsilon, b, bb, \ldots}=\set{b^n}{n\in N_0}\\
	[a]&=\set{b^na^m}{n\in\N_0, m\in\N^+}\\
	[ab]&=\set{w\in\simpleset{a,b}^\star}{w \text{ enthält das Teilwort $ab$ aber nicht das Teilwort $abb$}}\\
	[abb]&=\set{wabbw'}{w,w'\in\simpleset{a,b}^\star}
\end{align*}


\section{Pumping-Lemma für Typ-2}
\begin{itemize}
	\item
	Für die Sprache $L=\set{a^nb^nc^n}{n\geq 1}$:

	Zunächst wählt man ein Wort $x$, mit $|x|\geq n$:
	\begin{equation*}
		x=a^nb^nc^n \in L,\quad |a^nb^nc^n|=3n>n
	\end{equation*}
	Nun zu einer beliebigen Zerlegung $x=uvwxy$, für die die Bedingungen gelten. Da $|vwx|\leq n$, können $v$ und $x$ nur maximal zwei unterschiedliche Buchstaben beinhalten, niemals jedoch $a$, $b$ und $c$.

	Damit kann $uv^iwx^iy$ nicht in $L$ sein für ein $i\neq 1$. \hfill $\Box$
	\item
	Für die Sprache $L=\set{a^n}{n \text{ ist eine Quadratzahl}}$:

	Zunächst wählt man ein Wort $x$, mit $|x|\geq n$:
	\begin{equation*}
		x=a^{n^2} \in L
	\end{equation*}
	Bei jeder Zerlegung $x=uvwxy$ sind nur $a$s in den Teilwörtern $v$ und $x$, die gepumpt werden.

	Betrachtet man die Länge $|vx|=r$, so gilt:
	\begin{equation*}
		|uv^iwx^iy|=n^2+r(i-1)
	\end{equation*}
	Insbesondere gilt für das Wort
	\begin{align*}
		x'&=uv^2wx^2y=a^s\\
		|x'|&=n^2+r=s
	\end{align*}
	Das hieße, $n^2+r$ müsste eine Quadratzahl sein damit $x'$ wiederum in $L$ läge.
  Für $r$ gilt aber die Ausgangsbedinung des Pumping-Lemmas!
	\begin{equation*}
		|vwx|=r+|w|\leq n
	\end{equation*}
	$s$ kann damit unmöglich eine Quadratzahl sein. \hfill $\Box$
\end{itemize}

\section{Beweis durch Abschlusseigenschaften}
Zu zeigen: $L=\set{b^na^m}{n,m\in\N_0, n\neq m}\not\in\reg$
\begin{align*}
	\intertext{Annahme:}
	&L\in\reg\\
	\intertext{Mit dem Abschluss gegen Komplement folgt}
	&\overline L\in\reg\\
	\intertext{Wegen Abschluss gegen Schnitt folgt dann auch}
	&\overline L \cap L(b^\star a ^\star) \in\reg\\
	\intertext{Jedoch gilt}
	&\overline L \cap L(b^\star a ^\star) = \set{b^na^n}{n\in \N_0} \in \dcfl \supsetneq \reg
	\intertext{Widerspruch! $L\not\in\reg$}
\end{align*}
