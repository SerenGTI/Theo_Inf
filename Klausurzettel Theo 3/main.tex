\documentclass[landscape, 8pt, a4paper]{extarticle}
\usepackage[ngerman]{babel}
\usepackage[utf8]{inputenc}
\usepackage{lmodern}
\usepackage[T1]{fontenc}
\renewcommand{\familydefault}{\sfdefault}

\usepackage{amsmath}
\usepackage[fleqn]{mathtools}
\usepackage{amssymb}
\usepackage{amsfonts}
\usepackage{latexsym}
\usepackage[landscape, margin=1cm]{geometry}
\usepackage{enumitem}
\usepackage{multicol}
\usepackage{fancyhdr}
\usepackage{xcolor}


\mathcode`\*="8000 %Sterne durch Malpunkte ersetzen
{\catcode`\*\active\gdef*{\cdot}}

\pagestyle{empty}

\setlength{\parindent}{0pt}

\newcommand{\ggT}{\texttt{ggT}}
\newcommand{\kgV}{\texttt{kgV}}


\setlist[2]{noitemsep}
\setitemize{noitemsep, leftmargin=8pt, itemindent=0pt, labelsep=3pt, labelwidth=0pt, labelindent=0pt}
\setlength{\belowdisplayskip}{0pt} \setlength{\belowdisplayshortskip}{0pt}
\setlength{\abovedisplayskip}{0pt} \setlength{\abovedisplayshortskip}{0pt}

\newcommand{\N}{\mathbb{N}}
\newcommand{\Z}{\mathbb{Z}}
\newcommand{\set}[2]{\ensuremath\left\{ #1 \,\middle|\, #2 \right\}}
\newcommand{\simpleset}[1]{\ensuremath\left\{ #1 \right\}}
\renewcommand{\O}{\mathcal O}

\begin{document}
\begin{multicols}{3}
	\textbf{Simon König 3344789 - Klausurzettel Theoretische Informatik 3}

	\section{Wichtige Algorithmen}
	\begin{itemize}
		\item 
	\end{itemize}

	\section{Laufzeitanalyse}

	\begin{itemize}
		\item \textbf{Mastertheorem 1} Für $a,b\in\N$, $b>1$ und eine Funktion $g:\N\rightarrow \N$ mit $g\in\Theta(n^c)$ gelte
		\begin{align*}
			t(1)&=g(1)\\
			t(n)&=a*t\left(\frac nb\right)+g(n)
		\end{align*}
		Dann gilt 
		\begin{equation*}
			t(n)\in\begin{cases}
			\Theta(n^c)&\text{falls }a<b^c\\
			\Theta(n^c\log n)&\text{falls }a=b^c\\
			\Theta(n^{\frac{\log a}{\log b}})&\text{falls }a>b^c\\
			\end{cases}
		\end{equation*}

		\item \textbf{Mastertheorem 2} Sei $r>0$ und die Zahlen $\alpha_1\geq 0$ für alle $i$ und erfüllen $\sum_{i=1}^r\alpha_i<1$.
		Wenn die Rekursive Funktion $t$ die Ungleichung
		\begin{equation*}
			t(n)\leq \left( \sum_{i=1}^r t(\lceil \alpha_i*n\rceil) \right)+c*n
		\end{equation*}
		für ein $c>0$ erfüllt, dann ist $t(n)\in\O(n)$.

		\item \textbf{Summenformeln}
		\begin{equation*}
			\sum_{i=0}^n x^i = \frac{1-x^{n+1}}{1-x}
		\end{equation*}

		\begin{equation*}
			?
		\end{equation*}
	\end{itemize}



	\section{Diskrete Strukturen}
	\begin{itemize}
		\item Halbgruppe: assoziativ
		\item Monoid: neutrales Element
		\item Gruppe: beidseitig Inverse
		\item Abelsche Gruppe: kommutativ
		\item Ring: $(R,+,0)$ ist abelsche Gruppe, $(R,*,1)$ ist Monoid, Distributivgesetze
		\item Körper: $(R\setminus\simpleset 0, *,1)$ ist Gruppe und $*$ ist kommutativ

		
		\item Eigenschaften einer Kongruenzrelation:%
		\begin{itemize}%
			\item Reflexivität: $a\sim a$
			\item Symmetrie: $a\sim b\Rightarrow b\sim a$
			\item Transitivität: $a\sim b, b\sim c\Rightarrow a\sim c$
			\item {\color{red} Verträglichkeit mit Abbildung?}
		\end{itemize}

	\end{itemize}


	\section{Modulare Arithmetik}
	\begin{itemize}
		\item Allgemein gilt
		\begin{equation*}
			\ggT(ca,cb)=c*\ggT(a,b)
		\end{equation*}
		\item \textbf{Lemma von Bezout}
		Für alle $m,n\in\Z$ existieren $a,b\in\Z$ so, dass
		\begin{equation*}
			\ggT(m,n)=am+bn
		\end{equation*}
		das heißt, der größte gemeinsame Teiler lässt sich als Linearkombination darstellen.
		\item \textbf{Fundamentalsatz der Arithmetik} Die Primfaktorzerlegung jeder natürlichen Zahl ist eindeutig.
		\item Die multiplikative Gruppe besteht aus den Elementen, die teilerfremd zum Modul sind
		\begin{equation*}
			(\Z/n\Z)^\ast =\set{k+n\Z}{\ggT(k,n)=1}
		\end{equation*}
		\item $\Z/n\Z$ ist ein Körper genau dann, wenn $n$ eine Primzahl ist.

		\item Die lineare Abbildung $x\mapsto kx$ auf $\Z/n\Z$ ist genau dann bijektiv, wenn $\ggT(k,n)=1$.
		
		\item Sind $m,n$ teilerfremd, d.h. $\ggT(m,n)=1$, dann ist
		\begin{equation*}
			\pi:\Z\rightarrow\Z/m\Z\times\Z/n\Z, x\mapsto(x+m\Z,x+n\Z)
		\end{equation*}
		surjektiv. Damit erhält man eine bijektive Abbildung
		\begin{equation*}
			(x\mod mn)\mapsto (x\mod m,x\mod n)
		\end{equation*}
		\item \textbf{Chinesischer Restsatz}
		Für teilerfremde Zahlen $m,n$ ist die Abbildung
		\begin{equation*}
			\Z/mn\Z\rightarrow \Z/m\Z\times \Z/n\Z, 
			x+mn\Z\mapsto (x+m\Z,x+n\Z)
		\end{equation*}
		ein Isomorphismus (bijektiver Homomorphismus).

		\item \textbf{Der kleine Satz von Fermat} Für alle Primzahlen $p$ und alle $a\in\Z$ gilt
		\begin{equation*}
			a^p\equiv a\mod p
		\end{equation*}
		Falls $a$ und $p$ teilerfremd sind, gilt sogar
		\begin{equation*}
			a^{p-1}\equiv 1\mod p
		\end{equation*}

		\item \textbf{Satz von Euler} Für teilerfremde ganze Zahlen $a,n$ gilt
		\begin{equation*}
			a^{\varphi(n)}\equiv 1 \mod n
		\end{equation*}
		Verallgemeinert man den Satz (die multiplikative Gruppe $(\Z/n\Z)^\ast$ ist kommutativ), erhält man den \textbf{Satz von Lagrange} Es gilt sogar für jede kommutative Gruppe $G$ und jedes $a\in G$ 
		\begin{equation*}
		 	a^{|G|}=1
		\end{equation*} 

		\item Summe über die $\varphi(t)$ der Teiler von $n$ ist gleich $n$
		\begin{equation*}
			\sum_{t|n}\varphi(t)=n
		\end{equation*}
	\end{itemize}

\end{multicols}
\end{document}
