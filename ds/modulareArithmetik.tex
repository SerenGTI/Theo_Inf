%!TEX root = ../main.tex
\chapter{Modulare Arithmetik}
Wir rechnen in den sogenannten Restklassen. Für ein $n\in\N$ seien für $k\in\Z$ die Mengen
\begin{equation*}
	k+n\Z=\simpleset{\ldots,k-2n,k-n,k,k+n,k+2n,\ldots}.
\end{equation*}
die sogenannten Restklassen. Wir definieren hierauf die Äquivalenzrelation $\equiv (\operatorname{mod} n)$ durch
\begin{equation*}
	k\equiv l\mod n\Leftrightarrow k\in l+n\Z
\end{equation*}
Aus den Restklassen bilden wir den Restklassenring $\Z/n\Z$ mit den Verknüpfungen
\begin{align*}
	(k+n\Z)+(l+n\Z)&=k+l+n\Z\\
	(k+n\Z)*(l+n\Z)&=k*l+n\Z\\
\end{align*}
Die multiplikative Gruppe in $\Z/n\Z$ besteht aus den bezüglich Multiplikation invertierbaren Elementen. Sie wird bezeichnet mit $(\Z/n\Z)^\ast$.

Wir bezeichnen mit $\varphi(n)$ die Anzahl der natürlichen Zahlen $k<n$ für die $\ggT(k,n)=1$ gilt. Die multiplikative Gruppe $(\Z/n\Z)^\ast$ hat also $\varphi(n)$ viele Elemente.


\section{Wichtige Sätze und Lemmas}
\begin{itemize}
	\item Allgemein gilt
	\begin{equation*}
		\ggT(ca,cb)=c*\ggT(a,b)
	\end{equation*}
	\item \textbf{Lemma von Bezout}
	Für alle $m,n\in\Z$ existieren $a,b\in\Z$ so, dass
	\begin{equation*}
		\ggT(m,n)=am+bn
	\end{equation*}
	das heißt, der größte gemeinsame Teiler lässt sich als Linearkombination darstellen.
	\item \textbf{Fundamentalsatz der Arithmetik} Sei $n\in\N$. Dann lässt sich $n$ eindeutig darstellen als
	\begin{equation*}
		n=\prod_{p\text{ Prim}}  p^{n_p}
	\end{equation*}
	Dabei ist $n_p\neq 0$ genau dann, wenn $p$ ein Teiler von $n$ ist. Das heißt also, die Primfaktorzerlegung einer Zahl ist eindeutig.
	\item Die multiplikative Gruppe besteht aus den Elementen, die teilerfremd zum Modul sind
	\begin{equation*}
		(\Z/n\Z)^\ast =\set{k+n\Z}{\ggT(k,n)=1}
	\end{equation*}
	\item $\Z/n\Z$ ist ein Körper genau dann, wenn $n$ eine Primzahl ist.

	\item Die lineare Abbildung $x\mapsto kx$ auf $\Z/n\Z$ ist genau dann bijektiv, wenn $\ggT(k,n)=1$.
	
	\item Sind $m,n$ teilerfremd, d.h. $\ggT(m,n)=1$, dann ist
	\begin{equation*}
		\pi:\Z\rightarrow\Z/m\Z\times\Z/n\Z, x\mapsto(x+m\Z,x+n\Z)
	\end{equation*}
	surjektiv. Damit erhält man eine bijektive Abbildung
	\begin{equation*}
		(x\mod mn)\mapsto (x\mod m,x\mod n)
	\end{equation*}
	\item \textbf{Chinesischer Restsatz}
	Für teilerfremde Zahlen $m,n$ ist die Abbildung
	\begin{equation*}
		\Z/mn\Z\rightarrow \Z/m\Z\times \Z/n\Z
		x+mn\Z\mapsto (x+m\Z,x+n\Z)
	\end{equation*}
	ein Isomorphismus (bijektiver Homomorphismus).

	\item \textbf{Der kleine Satz von Fermat} Für alle Primzahlen $p$ und alle $a\in\Z$ gilt
	\begin{equation*}
		a^p\equiv a\mod p
	\end{equation*}
	Falls $a$ und $p$ teilerfremd sind, gilt sogar
	\begin{equation*}
		a^{p-1}\equiv 1\mod p
	\end{equation*}

	\item \textbf{Satz von Euler} Für teilerfremde ganze Zahlen $a,n$ gilt
	\begin{equation*}
		a^{\varphi(n)}\equiv 1 \mod n
	\end{equation*}

	dies gilt sogar für jede kommutative Gruppe $G$ und jedes $a\in G$ (die multiplikative Gruppe $(Z/n\Z)^\ast$ ist kommutativ)
	\begin{equation*}
	 	a^{|G|}=1
	\end{equation*} 

	\item Summe über die $\phi(t)$ der Teiler von $n$ ist gleich $n$
	\begin{equation*}
		\sum_{t|n}\varphi(t)=n
	\end{equation*}
\end{itemize}


\section{Beispiele und Anwendungen}
\subsection{Lösen von Kongruenzen}
\subsubsection{Finden der Inversen in der multiplikativen Gruppe}
Das Finden des Inversen eines Elements in der multiplikativen Gruppe $(\Z/n\Z)^\ast$ efolgt durch Lösen einer besonderen linearen diophantischen Gleichung. Wollen wir das Inverse von $a$ bestimmen, so wollen wir die Kongruenz
\begin{equation*}
	a*x\equiv 1 \mod n
\end{equation*}
für $x$ lösen. Das Inverse existiert nur, wenn $a$ und $n$ teilerfremd sind, das heißt $\ggT(a,n)=1$ ist. Damit lässt sich das Problem auf die diophantische Gleichung
\begin{equation*}
	a*x+n*y = 1 = \ggT(a,n)
\end{equation*}
reduzieren. Wir wissen aus dem Lemma von Bezout, dass sich der $\ggT$ als Linearkombination darstellen lässt. Diese können wir mit dem erweiterten euklidischen Algorithmus finden.
\paragraph{Beispiel:}
Wir wollen das Inverse von $17$ in $(\Z/31\Z)^\ast$ bestimmen. Wir wissen, dass $\ggT(17,31)=1$, wir wollen also folgendes lösen
\begin{align*}
	&17*x\equiv 1 \mod 31\\
	&17*x+31*y=1=\ggT(17,31)
\end{align*}
Dafür führen wir den euklidischen Algorithmus durch, der $\ggT$ ist markiert
\begin{align*}
	31&=1*17+14\\
	17&=1*14+3\\
	14&=4*3+2\\
	3&=1*2+\fbox{1} \leftarrow \text{ ab hier nach oben arbeiten.}\\
	2&=2*\fbox{1}+0.
\end{align*}
Gehen wir nun aus der vorletzten Zeile die Schritte rückwärts zurück, erhalten wir
\begin{align*}
	1&=3-2\\
	&=3-(14-4*3)=5*3-14\\
	&=5*(17-14)-14=5*17-6*14\\
	1&=5*17-6*(31-17)=\fbox{11}*17-6*31
\end{align*}
womit wir bereits die Linearfaktorzerlegung und damit insbesondere das Inverse gefunden haben.

\subsubsection{Lösen einer allgemeinen diophantischen Gleichung}
Möchte man eine lineare diophantische Gleichung 
\begin{equation*}
	a*x+b*y=c
\end{equation*}
lösen bei der $c\neq\ggT(a,b)$ ist, muss man wie folgt vorgehen.

Zu erst ist wichtig, dass $c$ ein Vielfaches von $\ggT(a,b)$ sein muss, sonst ist die Gleichung nicht lösbar.

Dann kann man wie oben vorgehen, den $\ggT$ berechnen und anschließend die vorletzte Zeile so erweitern, dass der Rest gleich $c$ ist. Verfährt man nun weiter wie oben und setzt rückwärts ein, erhält man die gesuchte Lösung.

\subsection{Fehlererkennung}


\subsection{Primzahltest und -Zertifikat}
\subsubsection{Primzahltest nach Fermat 22.3}

\subsubsection{Exaktes Primzahlzertifikat}
Sei $n\geq 2$ und $n\in\N$. Falls für alle Primzahlen $p$ mit $n\equiv 1\mod p$ eine Zahl $a\in Z$ existiert, so dass
\begin{equation*}
 	a^{n-1}\equiv 1\mod n \text{\quad und \quad}a^{\frac{n-1}p}\not\equiv 1\mod n
\end{equation*}
gilt, dann ist $n$ eine Primzahl. 

\subsection{Simultane Kongruenzen lösen}
\subsection{Schnelle Exponentiation 22.6}
\subsection{RSA}
Das RSA-Verfahren ist ein asymmetrisches Verschlüsselungsverfahren, das sehr weit verbreitet ist.

Möchte Bob  ($B$) eine Nachricht an Alice ($A$) senden, so muss Alice zunächst das folgende vorbereiten 
\begin{enumerate}
	\item Große Primzahlen $p,q$ mit $p<q$, beide müssen geheim bleiben.
	\item Berechne $n=p*q$ (öffentlich),
	\item setze $\varphi(n)=(p-1)(q-1)$,
	\item wähle $e>1$ mit $\ggT(e,\varphi(n)=1$ (öffentlich), diese Zahl wird zum Verschlüsseln verwendet.
	\item Berechne $s<n$ mit $e*s\equiv 1\mod \varphi(n)$, d.h. $e*s=k*\varphi(n)+1$. Diese Zahl wird zum Entschlüsseln verwendet und muss unbedingt geheim bleiben.
\end{enumerate}
Das Paar $(n,e)$ bildet den \emph{öffentlichen Schlüssel}.

Bobs Nachricht muss im Bereich $\simpleset{0,\ldots,n-1}$ sein, dies kann zum Beispiel durch Aufteilen der Nachricht in mehrere Teile erreicht werden. Die Nachricht sei $x$.

Bob verschlüsselt nun
\begin{equation*}
	y=x^e\mod n
\end{equation*}

Alice kann dann wieder entschlüsseln mit
\begin{equation*}
	x'=y^s\mod n
\end{equation*}

Dies funktionert, da
\begin{align*}
	(x^e)^s = x^{e*s}=x^{k*\varphi(n)+1}=x*(x^{\varphi(n)})^k\equiv x*1^k\mod n
\end{align*}