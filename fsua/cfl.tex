\chapter{Kontextfreie Sprachen $\cfl$: Typ-2}
\begin{equation*}
	\cfl\subset\csl\subset\rec\subset\re
\end{equation*}
Wort- und Leerheitsproblem entscheidbar.

Beschreib- und erkennbar durch einen nichtdeterministischen Kellerautomaten (PDA).
\section{Automatenmodell PDA:}


\section{Sätze zu den kontextfreien Sprachen:}
\begin{itemize}
	\item Jede kontextfreie Sprache über einem unären Alphabet ist regulär!
	\item Die Klasse der kontextfreien Sprachen ist abgeschlossen unter Sternoperation, Vereinigung und Konkatenation.
	\item Das Wortproblem ($\mathcal O(n^3)$) sowie das Leerheitsproblem sind entscheidbar.
\end{itemize}
\subsection{Pumping-Lemma für Typ-2:}
Sei $L\subseteq \Sigma^\star$ eine kontextfreie Sprache, dann gibt es eine Zahl $n$ so, dass für alle $z\in L$ mit $|z|\geq n$ eine Zerlegung mit $z=uvwxy$ in $u,v,w,x,y\in\Sigma^\star$ exisitert für die die drei Bedingungen erfüllt sind:
\begin{itemize}
	\item $|vx|\geq 1$
	\item $|vwx|\leq n$
	\item $\forall i\in N: uv^iwx^iy\in L$
\end{itemize}


\section{Chomsky-Normalform}
Eine Typ-2 Grammatik $(V,\Sigma,P,S)$ ist in Chomsky-Normalform (CNF),  wenn gilt:
\begin{equation*}
	\forall (u,v)\in P: v\in V^2\cup \Sigma
\end{equation*}
Zu jeder Typ-2 Grammatik existiert eine Grammatik $G'$ in CNF für die gilt $L(G)=L(G')$!
\subsection{Umformungsalgorithmus:}
\begin{enumerate}
	\item Zunächst wollen wir erreichen, dass folgendes gilt: $(u,v)\in P\Rightarrow (|v|>1 \vee v\in \Sigma)$
	\begin{enumerate}
		\item \textbf{Ringableitungen entfernen:}

		Eine Ringableitung liegt vor, wenn es Variablen $A_1,\ldots A_r$ gibt, die sich im Kreis in einander ableiten lassen, d.h. es gibt Regeln $A_i\rightarrow A_{i+1}$ und $A_r\rightarrow A_1$.

		Um dies loszuwerden, werden alle Variablen $A_i$ durch eine neue Variable $A$ ersetzt. Überflüssige Regeln wie $A\rightarrow A$ werden gelöscht.
		\item \textbf{Variablen anordnen:}

		Man legt eine Ordnung der Variablen fest: $V=\simpleset{B_1, B_2, \ldots, B_n}$, hierfür muss gelten:
		\begin{equation*}
			A_i\rightarrow A_j \in P \Leftrightarrow i<j
		\end{equation*}
		Falls dies nicht gilt, müssen Abkürzungen verwendet werden, also alle Produktionen von $A_j$ werden eingesetzt:
		\begin{equation*}
			P=(P\setminus\simpleset{A_i\rightarrow A_j})\cup\set{(A_i,w)}{(A_j,w)\in P}
		\end{equation*}
	\end{enumerate}
	\item Jetzt gilt für jede Regel $(u,v)\in P$ entweder $v\in\Sigma$ oder $|v|\geq 2$.

	Für letztere Regeln werden nun Pseudoterminale eingeführt. Es werden neue Variablen und Produktionen für jedes Terminalsymbol hinzugefügt, z.B. $V_a\rightarrow a$.

	\item \textbf{Letzer Schritt:} Alle rechten Seiten mit $|v|>2$ müssen nun noch auf Länge 2 gekürzt werden. Hierfür werden wiederum neue Variablen eingefügt:%
	\begin{align*}
		A\rightarrow C_1C_2C_3\\
		\intertext{Wird gekürzt zu}
		A\rightarrow C_1D_1\\
		D_1\rightarrow C_2C_3
	\end{align*}
\end{enumerate}

\section{Greibach-Normalform}
Eine Typ-2 Grammatik $(V,\Sigma,P,S)$ ist in Greibach-Normalform (GNF),  wenn gilt:
\begin{equation*}
	\forall (u,v)\in P: v\in \Sigma V^\star
\end{equation*}
Zu jeder Typ-2 Grammatik existiert eine Grammatik $G'$ in GNF für die gilt $L(G)=L(G')$!
\subsection{Umformungsalgorithmus:}
\begin{enumerate}
	\item \textbf{Mh?}

	\item \textbf{Beseitigung von Linksrekursion:}

	Alle Produktionsregeln sind von der Form:
	\begin{align*}
		A&\rightarrow A\alpha_1|\ldots|A\alpha_k|\beta_1|\ldots|\beta_l\\
		\intertext{Diese können durch diese $2k+2l$ Regeln ersetzt werden:}
		A&\rightarrow \beta_1|\ldots|\beta_l\\
		A&\rightarrow \beta_1B|\ldots|\beta_lB\\
		B&\rightarrow \alpha_1|\ldots|\alpha_k\\
		B&\rightarrow \alpha_1B|\ldots|\alpha_kB
	\end{align*}%
	Nun sind keinerlei Linksrekursionen mehr vorhanden!
\end{enumerate}

\section{Beispiele: }
\begin{itemize}
	\item $L_1=\set{a^nb^n}{n\geq 1}$ (zwei gleiche Exponenten)
	\item $L_2=\set{ww^R}{w\in\Sigma^\star}$ (unmarkierte Palindrome)
	\item Korrekt geklammerte arithmetische Ausdrücke (Dyck-Wörter)
\end{itemize}
